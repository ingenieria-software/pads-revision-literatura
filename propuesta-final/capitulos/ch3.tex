\section{Tipo de Investigación}

\section{Fuentes y Sujetos de Investigación}

\subsection{Fuentes}

\subsubsection{Los planes de estudio de las universidades.} De forma más específica se pretende evaluar los programas de los cursos que involucren desarrollo y/o análisis de software de alguna forma. Preliminarmente, cursos tales como:
\begin{itemize}
    \item Programación (cursos introductorios y avanzados)
    \item Desarrollo Web
    \item Aseguramiento de la Calidad
    \item Análisis/Diseño de Sistemas
    \item Ingeniería de Software
    \item Ingeniería de Requerimientos
    \item Bases de Datos
\end{itemize}

se muestran como candidatos iniciales para el estudio. Se espera que durante la ejecución de la investigación los cursos a evaluar puedan variar de una universidad a otra, pero esto es esperado debido a que cada institución maneja énfasis de enseñana diferentes. El trabajo conjunto con las autoridades de la universidad hará que se logre identificar de forma más precisa cursos a evaluar dependiendo de la estructura del plan de estudios.

Este estudio no tomará en consideración cursos que no tenga relación alguna con software, como lo podrían ser cursos de matemática, física, arquitectura y redes de computadores, idiomas y de estudios generales.

\subsubsection{Personal académico y administrativo de las universidades.} 
Con ellos se pretende trabajar de forma conjunta para valorar la penetración de prácticas ágiles de desarrollo de software en los cursos relacionados con desarrollo y/o análisis de software. Mediante la aplicación de una encuesta y entrevistas presenciales es que se desea estimar la importancia que se le dan a estos temas dentro de sus planes de estudio así como el intercambio experiencias con los mismos, casos de éxito/fracaso y limitaciones que hayan encontrado.

\subsubsection{Informes del estado actual de prácticas ágiles de desarrollo de software}
El trabajo llevado a cabo en el capítulo \ref{ch2:marco-teorico}, así como los informes de referencia de la industria \cite{version-one, puppet-devops} brindan un punto de partida para saber qué es lo relevante actualmente y compararlo con lo hallado en los planes de estudio en cuestión.


\subsubsection{Población}
Los planes de estudio de carreras en computación, informática y sistemas de información listadas en el Cuadro \ref{table:listado-conesup}, así como personal académico y admistrativo a cargo de los mismos.

\subsubsection{Muestra}
Los planes de estudio de carreras en computación, informática y sistemas de información acreditadas por el SINAES, listadas en el Cuadro \ref{table:listado-sinaes}, así como personal académico y admistrativo a cargo de los mismos.

\subsection{Sujeto de la investigación}
Los planes de estudio de las universidades serán el principal sujeto de investigación de esta propuesta, específicamente en cómo incorporan prácticas ágiles de desarrollo en ellos.


\section{Técnicas de Investigación}
Para esta propuesta se propone una investigación de enfoque cualitativo. De acuerdo con  Creswell \cite{creswell} \emph{``la investigación cualitativa es un enfoque para explorar y entender el significado de individuos o grupos envueltos en un problema humano. El proceso de investigación involucra preguntas emergentes y procediientos, datos recolectados típicamente desde los participantes, análisis de datos de forma inductiva de temas particulares a temas generales y en donde el investigados hace interpretaciones del significado de los datos. El reporte final se tiene una estructura flexible. Aquellos que participan de esta forma de investigación apoyan una forma de ver la investigación que honra el estilo inductivo, un enfoque de significado invidual y la importancia de representar la complejidad de una situación''}.

De acuerdo a lo que ha venido exponiendo a lo largo de esta propuesta, este enfoque es el más adecuado debido a la naturaleza exploratoria y de interacción humana. Las siguientes técnicas de investigación 


\section{Procesamiento y Análisis de Datos}