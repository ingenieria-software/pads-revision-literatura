\section{Tipo de Investigación}
Debido a que en esta propuesta de investigación se contempla la elaboración de encuestas y entrevistas para la recolección y análisis de datos, se va a hacer uso de un enfoque cualitativo.


\section{Fuentes y Sujetos de Investigación}

\subsection{Fuentes}

\subsubsection{Los planes de estudio de las universidades.} De forma más específica se pretende evaluar los programas de los cursos que involucren desarrollo y/o análisis de software de alguna forma. Preliminarmente, cursos tales como:
\begin{itemize}
    \item Programación (cursos introductorios y avanzados)
    \item Desarrollo Web
    \item Aseguramiento de la Calidad
    \item Análisis/Diseño de Sistemas
    \item Ingeniería de Software
    \item Ingeniería de Requerimientos
    \item Bases de Datos
\end{itemize}

se muestran como candidatos iniciales para el estudio. Se espera que durante la ejecución de la investigación los cursos a evaluar puedan variar de una universidad a otra, pero esto es esperado debido a que cada institución maneja énfasis de enseñana diferentes. El trabajo conjunto con las autoridades de la universidad hará que se logre identificar de forma más precisa cursos a evaluar dependiendo de la estructura del plan de estudios.

Este estudio no tomará en consideración cursos que no tenga relación alguna con software, como lo podrían ser cursos de matemática, física, arquitectura y redes de computadores, idiomas y de estudios generales.

\subsubsection{Personal académico y administrativo de las universidades.} 
Con ellos se pretende trabajar de forma conjunta para valorar la penetración de prácticas ágiles de desarrollo de software en los cursos relacionados con desarrollo y/o análisis de software. Mediante la aplicación de una encuesta y entrevistas presenciales es que se desea estimar la importancia que se le dan a estos temas dentro de sus planes de estudio así como el intercambio experiencias con los mismos, casos de éxito/fracaso y limitaciones que hayan encontrado.

\subsubsection{Informes del estado actual de prácticas ágiles de desarrollo de software}
El trabajo llevado a cabo en el capítulo \ref{ch2:marco-teorico}, así como los informes de referencia de la industria \cite{version-one, puppet-devops} brindan un punto de partida para saber qué es lo relevante actualmente y compararlo con lo hallado en los planes de estudio en cuestión.


\subsubsection{Población}
Los planes de estudio de carreras en computación, informática y sistemas de información listadas en el Cuadro \ref{table:listado-conesup}, así como personal académico y admistrativo a cargo de los mismos.

\subsubsection{Muestra}
Los planes de estudio de carreras en computación, informática y sistemas de información acreditadas por el SINAES, listadas en el Cuadro \ref{table:listado-sinaes}, así como personal académico y admistrativo a cargo de los mismos.

\subsection{Sujeto de la investigación}
Los planes de estudio de las universidades serán el principal sujeto de investigación de esta propuesta, específicamente en cómo incorporan prácticas ágiles de desarrollo en ellos.


\section{Técnicas de Investigación}
Para esta propuesta se propone una investigación de enfoque cualitativo. De acuerdo con  Creswell \cite{creswell} \emph{``la investigación cualitativa es un enfoque para explorar y entender el significado de individuos o grupos envueltos en un problema humano. El proceso de investigación involucra preguntas emergentes y procediientos, datos recolectados típicamente desde los participantes, análisis de datos de forma inductiva de temas particulares a temas generales y en donde el investigados hace interpretaciones del significado de los datos. El reporte final se tiene una estructura flexible. Aquellos que participan de esta forma de investigación apoyan una forma de ver la investigación que honra el estilo inductivo, un enfoque de significado invidual y la importancia de representar la complejidad de una situación''}.

De acuerdo a lo que ha venido exponiendo a lo largo de esta propuesta, este enfoque es el más adecuado debido a la naturaleza exploratoria y de interacción humana. Para la realización de esta investigación se proponen las siguientes técnicas de investigación:
\begin{itemize}
    \item Encuesta: para evaluar el nivel de adopción e importancia que se le da a las prácticas ágiles de desarrollo de software tanto en los planes de estudio como también desde el punto de vista del personal académico y administrativo de la universidad. 
    \item Entrevista: para discutir tanto los resultados de la encuesta con el personal de la universidad y también como puntos de vista, experiencias y expectativas con respecto al tema.   
\end{itemize}
 
\subsection{Encuesta}
Debido a lo que se planea en primera instancia es explorar el nivel de penetración de las prácticas, se ha confeccionado una encuesta con 5 preguntas para obtener apreciaciones iniciales del personal a cargo de la universidad.

El detalle de la encuenta se encuentra en el apéndice 1.


\subsection{Entrevista}

Como preparación para la entrevista, se solicitará de antemano una copia del plan de estudios de la universidad y los programas de los cursos que para el investigador tenga relación directa con desarrollo de software. De esta forma se podrán analizar estos documentos y contrastar la información contenida en ellos con los entrevistados.

\begin{enumerate}
    \item ¿Cuál es el perfil profesional del estudiante graduado de esta carrera?
    \item ¿Qué objetivos buscaría la universidad con la enseñanza de estas prácticas?
    \item ¿Qué otros efectos se considera que podría causar la adopción activa de estas prácticas?
    \item Debido a que mucho del uso que se hace de estas prácticas se da en ambientes en la nube, ¿se expone a los estudiantes a tecnólogías en la nube durante su paso por la carrera?
    \item Si se tiene planeado adoptar la enseñanza de estas prácticas, ¿cuáles prácticas se tienen en consideración?
    \item ¿Los estudiantes han externalizado su interés en la formación estas prácticas?
    \item ¿La universidad tiene algún tipo convenio con empresas con el fin de dar a conocer y/o entrenar a estudiantes en estos temas?
\end{enumerate}



\section{Procesamiento y Análisis de Datos}
Se propone seguir el enfoque de análisis cualitativo de \cite{runeson-et-al} para esta tarea, de acuerdo con este enfoque \emph{``La idea es identificar generalizaciones en los datos términos de patrones, sequencias, relaciones, entre otros. El análisis se lleva a cabo de forma iterativa, esto significa que se para a través del material varias veces y cada vez se identifica o ajusta códigos, categorías y similares. Es importante ser sistemático y documentar cada paso de la investigacio4n. Esto hace posible describir una ``cadena de evidencia'' para el lector in el cual no hay forma de que el lector entienda que las conclusiones son confiables''}.


\subsection{Pasos en el análisis}

\paragraph{1.} Estudiar el material recolectado en detalle, incluso si ya fue estudiado durante las entrevistas. 

\paragraph{2.} Formular un conjunto de códigos de interés. Es usualmente beneficioso hacer esta formulación como una actividad de grupo. Códigos candidatos:
\begin{center}
\begin{itemize}
    \item PV: punto de vista
    \item EXP: experiencia en la docencia
    \item BEN: beneficios y objetivos
    \item LIM: limitación
    \item FUT: planes futuros
\end{itemize}

\end{center}


\paragraph{3.} Leer todos los textos y marcar en donde los códigos encajen con los contenidos.\\
\textbf{3.1.} Transcribir las entrevistas a archivos de texto y marcar secciones de interés\\
\textbf{3.2.} Asignar códigos a los textos marcados \\
\textbf{3.3.} Tabular los resultados de la encuesta. \\
\textbf{3.4.} Aplicar técnicas de estadística descriptiva a los resultados de la encuesta.

\paragraph{4.} Utilizar el material codificado para llegar a conclusiones. Comparar el texto para diferentes códigos. Por ejemplo, si hay un tipo de tipo de enunciado que es común para todos los códigos de un tipo pero no para otro, incluso si podría esperarse también para el otro tipo, entonces eso se puede investigar más de cerca.

\paragraph{5.} Búsqueda de patrones. Con este enfoque un patrón empírico es observado y comparado con un patrón predicado. Cuando hay coincidencia, esto apoya a que el predicado sea verdadero. La búsqueda de patrones se puede basar en explicaciones rivales. Si tiene varios casos con el mismo resultado, es posible formular un conjunto de explicaciones rivales para este resultado. Entonces es posible investigar los casos y determinar cuál de las explicaciones que describen los casos de la mejor manera.

\paragraph{5.1.} Construcción de una explicación. Los patrones son identificados basados en una relación causa-efecto y se buscan explicaciones subyacentes. 

\paragraph{5.2.} Estudio de casos similares. Este tipo de análisis puede ser conducido si se tienen otros casos de referencia. En este caso se puede utilizar el material recolectado en la Sección \ref{ch2:marco-teorico} para realizar comparaciones.


\paragraph{7.} Validación de pares

\paragraph{6.} Generar un reporte el que se de a conocer lo encontrado.
 