\section{Revisión de la Literatura}

\subsection{Estrategia} \label{sec:estrategia}
Esta sección presenta la estrategia emprendida para cubrir el cuerpo de conocimiento relacionado con enseñanza de prácticas ágiles de desarrollo de software en universidades. La estrategia general se basó en un proceso iterativo de identificación y lectura de artículos, luego identificar y leer artículos relevantes a partir de referencias y citas bibliográficas.

\subsection{Identificación de Preguntas de Investigación}
Las preguntas de investigación seleccionadas para conducir la revisión de la literatura fueron:
\begin{enumerate}
    \item ¿Cuáles son las prácticas de desarrollo ágil con mayor aceptación y qué relevancia tienen en la industria?
    \item ¿Se podría considerar a DevOps como una práctica de desarrollo ágil o representa más bien un enfoque diferente?
    \item ¿Qué información hay disponible acerca de de la enseñanza de prácticas ágiles de desarrollo de software carreras universitarias de TIC's en Costa Rica como en el extranjero? 
    \item ¿Cuáles son los beneficios y retos reportados en la enseñanza de estas prácticas de desarrollo ágil?
\end{enumerate}

\subsection{Estrategia de búsqueda} \label{sec:estrategia-busqueda}
Con el fin de identificar el primer conjunto de artículos relevantes, se hizo una revisión preliminar con Google Scholar\footnote{\url{http://scholar.google.com}} porque con este motor de búsqueda se puede abarcar una amplio número de artículos y actas académicas de diferentes fuentes. El criterio de búsqueda se basó en búsquedas de palabras derivadas del tema de investigación y las preguntas de investigación. Se incluyeron palabras como ``agile development'', ``agile engineering practices'', ``agile teaching'', ``devops education'', ``computer science education'' y ``software engineering education''. 

La búsqueda de la literatura se realizó en Setiembre del 2017 usando las siguientes bases de datos electrónicas:
\begin{itemize}
    \item ACM \emph{Digital Library} 
    \item IEEE \emph{Explore Digital Library}
    \item Safari Books Online\footnote{\url{https://www.safaribooksonline.com}}
    \item Google Scholar
\end{itemize}

\subsection{Criterio de selección de artículos}
Se aplicó el siguiente criterio de inclusión de artículos para esta revisión:
\begin{itemize}
    \item Estudios que den a conocer prácticas ágiles de desarrollo de software y 
    \item Estudios sobre la relación de prácticas ágiles de desarrollo de software y DevOps 
    \item Estudios que proporcionen algún tipo de solución, guía o marco de trabajo relacionado con la enseñana de prácticas ágiles de desarrollo de software en las universidades
    \item Estudios que reporten sobre el éxito, fracaso y retos de la experiencia de la enseñanza de prácticas ágiles de desarrollo de software
    \item Estudios que proporcionen evidencia sobre la enseñanza de prácticas ágiles de desarrollo de software y el impacto en la industria
\end{itemize}

El criterio de exclusión de artículos fue el siguiente:
\begin{itemize}
    \item Estudios relacionados en dar a conocer aspectos sobre procesos operativos y de negocios asociados con metodologías de desarrollo ágil.
    \item Estudios relacionados en la enseñanza de metodologías ágiles pero que no cubren o cubren muy poco aspectos sobre prácticas ágiles de desarrollo de software.
    \item Estudios relacionados con casos de estudio de la aplicación y/o enseñanza de prácticas ágiles de desarrollo de software fuera de la universidad.
    \item Artículos publicados hace más de 10 años atrás (rango aceptable 2007-2017).
\end{itemize}


\subsection{Resultado de la revisión} \label{sec:resultado-rev-lit}
Luego de obtener los artículos, literatura y recursos relevantes, se identificó que los mismos podían clasificarse en tres grupos: desarrollo ágil de software, enseñanza de prácticas ágiles de desarrollo de software en universidades extranjeras y enseñanza de prácticas ágiles de desarrollo de software en Costa Rica.


%¿Cuál es el nivel de penetración de prácticas ágiles de desarrollo de software en los planes de estudio de carreras en tecnología de la información (TIC's) en Costa Rica y qué factores propician o limitan esto?

\section{Desarrollo Ágil de Software} \label{sec:desarrollo-agil}
Las metodologías de desarrollo ágil de software emergen al final de de la década de los noventa. El término ``ágil'' se utiliza para agrupar una serie de métodos tales como \emph{Scrum}, \emph{eXtreme Programming}, Crystal, \emph{Feature Driven Development} (FDD), \emph{Dynamic Software Development Method} (DSDM) y \emph{Adaptative Software Development}\cite{rashina-et-al}. Las metodologías ágiles se caracterizan por promover el desarrollo interativo e incremental y la entrega frecuente de funcionalidad prioritaria para los clientes. Estas metodologías estan dirigidas a equipos pequeños y altamente colaborativos. \emph{Scrum} y XP son las metodologías que presentan mayores niveles de adopción\cite{version-one}: XP se centra en prácticas de desarrollo mientras que \emph{Scrum} cubre principalmente la administración del proyecto.

Las metodologías ágiles son adecuadas para proyectos con requerimientos altamente cambiantes, se motiva aceptar y responder ante los cambios. En 2001, los desarrolladores de varios de estas metodologías ágiles escribieron el Manifiesto Ágil \cite{agile-manifesto} (Apéndice \ref{apendice:a}). Los principios detrás del Manifiesto Ágil incluyen entrega rápida, frecuente consistente y continua de software funcional, respuesta a requerimientos cambiantes, comunicación efectiva y equipos motivados con capacidad de auto organizarse.

\subsection{Prácticas Ágiles de desarrollo de software} \label{sec:practicas-agiles-desarrollo}
De acuerdo con \cite{ford}, a pesar de la gran cantidad de literatura disponible sobre metodologías ágiles de software, mucha de la misma menciona poco acerca de las prácticas agiles de desarrollo de software. La mayoría se centran principalmente en gestión, procesos y estimación, y no tanto en la parte relacionada con la ingeniería de software.

En contraste con las metodologías, las prácticas ágiles están un nivel por debajo debido a estas ya que son una parte muy específica de una metodología, la cual aborda varios aspectos. Algunos ejemplos conocidos son programación en parejas, desarrollo orientado en pruebas y revisiones de código. A pesar de no haber una definición común en la literatura de práctica ágil \cite{diebold-dahlem}, se puede tomar XP en consideración para tener un mejor entendimiento por medio del estudio de la colección de prácticas de desarrollo que se promueven en esta metodología. A diferencia de otras metodologías como \emph{Scrum}, XP se dedica tanto a la gestión del proyecto como también en la forma en cómo los equipos construyen código \cite{shore-warden}.

Para efectos de esta revisión, se definen un conjunto de prácticas. Esto porque las metodologías ágiles llaman a sus prácticas de forma diferente. Para obtener un conjunto de prácticas de referencia, se tomó como punto de partida el reporte anual del estado de la metodologías ágiles de Version One \cite{version-one}, y el cual expone las prácticas ágiles en desarrollo de mayor aceptación. Las prácticas expuestas en este reporte se confrontaron con la literatura acerca de XP \cite{beck-andres, ford, shore-warden} con el fin de validar si se hace referencia de las mismas. El resultado es la siguiente lista de prácticas ágiles en desarrollo de software:
\begin{itemize}
    \item Pruebas
        \begin{itemize}
            \item Pruebas unitarias
            \item Desarrollo orientado a pruebas (TDD, por sus siglas en inglés)
            \item Desarrollo orientado al comportamiento (BDD, por sus siglas en inglés)
            \item Desarrollo orientado a pruebas de aceptación (ATDD, por sus siglas en inglés)
        \end{itemize}
        \item Integración continua
        \item Estándares de codificación
        \item Refactorización de código
        \item Puesta en producción continua
        \item Propiedad compartida del código
        \item Diseño emergente
\end{itemize}

En el apéndice \ref{apendice:b} se proveen definiciones de estas prácticas.

\subsection{DevOps}
DevOps es la práctica que combina desarrollo y operaciones y que desde el 2009 viene siendo adoptada por muchas organizaciones de la industria \cite{bang-et-al}. Se puede resumir como una práctica que anima a establecer un ambiente en donde la construcción, pruebas y la entrega de software puede realizarse de forma rápida, frecuente y más confiable \cite{henrik-b}. DevOps apareció como una respuesta directa a los retos de las plataformas de software a gran escala que se actualizan rápidamente, tal y como lo indica Anderson \cite{anderson}: \emph{``El movimiento DevOps emergió a partir de uno de los clásicos obstáculos en muchas organizaciones. Los desarrolladores construyen código y aplicaciones, y las envían al personal de operaciones solo para descubrir que el código y las aplicaciones no corren en producción. Es el clásico problema de ``se ejecutaba en mi máquina y funcionaba, ahora es problema del personal de operaciones'' ''}. 

\subsection{DevOps vs Metodologías Ágiles} \label{sec:devops-vs-agile}
A pesar de no estar considerada como una metología de desarrollo ágil como tal, autores como Bæebark \cite{henrik-b} ven al DevOps como el siguiente paso natural del movimento de desarrollo ágil el cual hace énfasis en software funcional, colaboración, velocidad y respuesta al cambio. Sin embargo, DevOps tiene mayor énfasis en las operaciones e introduce elementos nuevos: el código nuevo, arreglo de errores e incrementos son desarrollados, probados, integrados y entregados a los usuarios finales en cuestión de horas, y el equipo de desarrollo es responsable total de los requerimientos, desarrollo, pruebas, entrega y monitoreo. En \cite{jabbari-et-al} se estudió la relación de las prácticas de DevOps con otras metodologías de desarrollo y se destaca con respeto a DevOps y a las metodologías ágiles que:
\begin{itemize}
    \item DevOps extiende las metodologías ágiles: los principios de DevOps puede proporcionar una extensión prágmatica a las prácticas ágiles. Se pueden lograr los objetivos de las metodologías ágiles al extender los principios de estas metodologías a través de una tubería de entrega de software.
    \item Las metodologías ágiles son facilitadores para DevOps: en \cite{jabbari-et-al} se menciona que las metodologías ágiles pueden ser consideradas como facilitadores para adoptar los principios detrás de DevOps.
    \item Las metodologías ágiles apoyan DevOps: al motivar la colaboración entre los miembros de los equipos, la automatización de la construcción, entrega y pruebas, el establecimiento de métricas y el intercambio de conocimientos y herramientas. 
\end{itemize}

De igual forma, para validar esta relación e interoperatibilidad entre ambos enfoques, la búsqueda de literatura realizada en \ref{sec:estrategia-busqueda} en Safari \emph{Books Online} retornó   329 títulos los cuales incluían libros y videos. Esto sugiere que hay mucha aceptación en el uso de ambos enfoques en conjunto.

En \cite{jabbari-et-al} también se expone que a pesar de las similitudes, DevOps no cumple con todos los principios propuestos en el Manifiesto Ágil \cite{agile-manifesto} y que DevOps elimina la brecha entre desarrolladores y personal de operaciones mientras que las metodologías ágiles están orientadas a alinear requirimientos de negocio con el desarrollo.

De acuerdo con \cite{jabbari-et-al}, las principales prácticas en desarrollo de software de DevOps son:
\begin{itemize}
    \item El código como infraestructura
    \item Administración de la configuración
    \item Integración continua
    \item Entrega continua
    \item Pruebas automatizadas
    \item Monitoreo de rendimiento
\end{itemize}
La guía en \cite{wiggins} es un ejemplo de la aplicación de las prácticas de DevOps identificadas en \cite{jabbari-et-al} adaptadas para desarrollo de aplicaciones de tipo software como servicio o SAAS\footnote{\emph{Software-as-a-service}} por sus siglas en inglés.

Al igual que en la sección \ref{sec:practicas-agiles-desarrollo}, en el apéndice \ref{apendice:b} se proveen definiciones de estas prácticas.

\section{Relevancia de Prácticas ágiles de desarrollo} \label{sec:relevancia-agile}
Los reportes consultados sobre el estado actual de metodologías ágiles\cite{version-one} y DevOps\cite{puppet-devops} reflejan una amplia aceptación de ambos enfoques en la industria de tecnología. Por ejemplo, en \cite{version-one} el 94\% de las organizaciones que tomaron parte del estudio respondieron afirmativamente a la pregunta de si practicaban enfoques ágiles y el 71\% de las mismas reportó que estan llevando a cabo iniciativas en DevOps. En \cite{puppet-devops} se señala que para lograr que las personas logren el máximo provecho de DevOps, las prácticas técnicas tales como automatización de pruebas, puesta en producción automática, integración continua y un manejo de versiones detallado son claves. Los que práctican DevOps tienen que estar concientes del uso de estas prácticas técnicas y no solamente de herramientas específicas. 

La encuesta realizada en \cite{kropp-meier-1} donde 160 compañías suizas y casi 200 profesionales en tecnología de información participaron, también muestra resultados claros. Las compañías y los profesionales que siguen metodologías ágiles están más satisfechos con los enfoques que estas promueven que con metodologías basadas en planeamiento. Otro resultado importante del estudio es que los principales objetivos de introducir desarrollo ágil son: la gestión de prioridades cambiantes y la mejora del proceso de desarrollo en general. La encuesta muestra que no hay tantos ingenieros con habilidades para desarrollo ágil y esto le sugiere a los autores que los estudiantes no están siendo educados con las habilidades necesarias. El 70\% de los participantes piensan que los bachilleres en computación tiene muy poco conocimiento de metodologías ágiles; la mayoría piensa lo mismo para los graduados de maestría. Cuando se consultó sobre si las metodologías ágiles deberían ser parte integral del plan de estudios de carreras de computación, la gran mayoría recomendó que el desarrollo de software ágil debería de ser parte integral de estos planes de estudio.

Un estudio similar se llevó a cabo en el estado de Minnesota, Estados Unidos, con el fin de evaluar una nuevas opciones en los planes de estudio de carreras de tecnología de información de las universidades de ese estado \cite{advance-it}. El estudio señala que las habilidades en desarrollo ágil son cada vez más importantes en las decisiones de contratación, el 62\% lo considera un factor relevante y el 41\% comenzaron a hacerlo en los últimos 3 años. El 59\% dice que su desarrollo y entrega de software están basadas en prácticas ágiles/DevOps/\emph{Cloud Computing}. Hay preocupación sobre la disposición de la mano de obra de Minnesota para promover una \emph{transformación digital} y de la capacidad del sistema educacional del estado en producir esta fuerza de trabajo: solamente el 32\% está de acuerdo que la fuerza de trabajo está bien preparada en términos de competencias y el 49\% no están de acuerdo en que el sistema educativo está produciendo una fuerza laboral digital suficientemente grande.

Los resultados de la revisión sistemática de la literatura realizada por Rademarcher y Walia \cite{radermacher-walia} para determinar qué áreas de los estudiantes graduados con más frecuencia no alcanzan las expectativas de la industria o la academia indican que los estudiantes recién graduados presentan carencias en áreas técnicas, personales y profesionales. Dentro de las carencias en las áreas técnicas, el estudio de \cite{radermacher-walia} señala las siguientes:
\begin{itemize}
    \item Carencias en herramientas de software: una categoría que reporta muchas deficiencias y que muestra varios ejemplos específicos. El ejemplo más comunmente reportado es la gestión de la administración de la configuración. Los estudios consultados en este artículo encontraron que los desarrolladores de software recién contratados carecían de experiencia en herramientas de gestión de la configuración. Así mismo se mencion que existe una brecha grande entre las habilidades de los profesionales de software con respecto a su educación.
    \item Pruebas: es una área que presenta deficiencias tanto en la industria como en la academia. Estudiantes avanzandos mostraron poca habilidad de uso de herramientas para cobertura de de pruebas y en escribir pruebas relevantes. Los desarrolladores de software recién contratados muestran carencias en la habilidad de llevar a cabo pruebas de código exhaustivas. También se reportaron problemas con respeto al \emph{debugging} de código.
\end{itemize}


\subsubsection{En Costa Rica}
El sector de tecnología de la información en Costa Rica ha experimentado el mismo crecimiento explosivo con respecto a la oferta y demanda de personal calificado que se ha llegado a reportar en otros países. De 1997 al 2000 las compañías de desarrollo de software y otros emprendimientos crecieron a una tasa del 40\% al 60\% en la planilla\cite{cenfotec-2}.

De acuerdo con \cite{prosic} con respecto a la fuerza laboral total del país, el empleo en el sector de tecnologías de la información representó en 2006 un $2,32\%$ y en 2007, $2,47\%$, mostrando un leve aumento de $0.15\%$ para el último año. El aumento de $4886$ trabajadores en el sector de tecnologías de la información representa un crecimiento del $10.7\%$ en relación con la fuerta laboral todal del sector. Por otro lado, el comportamiento del empleo en el área de producción de bienes y servicios de las tecnologías de la información, entre 2006 y 2007 respecto de la fuerza laboral, fue inverso: el porcentaje de trabajadores dedicados a la producción disminuyó un $19.8\%$, mientras que el porcentaje de empleados dedicados a servicios aumentó un $19.6\%$.

En \cite{murillo-trejos} se hizo un mapeo de las habilidades consideradas las más importantes a encontrar en profesionales en tecnologías de la información. Una de las motivaciones de este mapeo es dar a conocer las áreas en las que las empresas multinacionales están más interesadas y de esta forma motivar la creación de nuevos programas educativos para estimular esas habilidades. Las categorías que reportaron mayor interés en las multinacionales fueron las relacionadas con ``Desarrollo e Implementación'' y ``Entrega y Operación''. Esto refleja que las hay un interés marcado en contar con profesionales que puedan tener habilidades que cubran todas las necesidades que se presenten durante el ciclo de desarrollo de sistemas.


El estudio llevado a cabo por el Programa Sociendad de la Información y el Conocimiento \cite{prosic}(Prosic) de la Universidad de Costa Rica señala que hay una preocupación en el país por la escasez de mano de obra calificada para el sector de las tecnologías de la información, particularmente en la industria del software. Se añade que \emph{``en el país no se han tomado las medidas necesarias para que este sector se consolide dentro de una economía del conocimiento. Esta consolidación requiere, entre otras cosas, mayores inversiones en investigación y desarrollo y
adecuar los sistemas educativos existentes en el país para que preparen profesionales en adecuada cantidad, así como con buena calidad para satisfacer los requerimientos del mercado laboral.''}

Dentro del mismo estudio, se aplicó una encuesta a 160 organizaciones de tecnologías de la información y se identificó que el principal problema de para el desarrollo del sector es la poca disponibilidad de recurso humano calificado. En este apartado la palabra ``calificado'' es motivo de diferenciación porque aunque se cuenta con fuerza laboral, en muchos casos se ha logrado experimentar insuficiencia en la búsqueda de los conocimientos que se desea que posean. Se acusa de un rezago educativo y recomiendan:
\begin{itemize}
    \item Fortalecimiento de la educación técnica
    \item Ajuste de los programas: actualización de formación y curricula
    \item Actualización de docentes
    \item Diseño de carreras efectivamente articuladas
    \item Integración de oferta y demanda
    \item Integración entre empresas y sector educativo para la definición de perfiles de formación
\end{itemize}


\section{Enseñanza de Prácticas Ágiles de Desarrollo de Software} \label{sec:ensenanza}
De acuerdo con los resultados obtenidos a partir la búsqueda en la sección \ref{sec:resultado-rev-lit}, se evidencia que la enseñanza de prácticas ágiles de desarrollo es un tema activo de investigación y desarrollo en universidades extranjeras. En \cite{kropp-meier-1} se señala que a pesar que el desarrollo de software ágil ha existido por más de una década, incluso antes del Manifiesto Ágil, la enseñanza de desarrollo de software ágil sólo ha llamado la atención en conferencias educativas y de investigación en años recientes. Una razón para esto puede ser que el desarrollo ágil no tiene bases teóricas sino que más bien ha sido desarrollada a partir de la práctica. En \cite{hazzan-dubinsky} se discute las razones por las cuáles los programas de ingeniería de software deberían de enseñar desarrollo ágil de software. Enfatizan que los ingenieros de software no solamente necesitan ejercitar habilidades técnicas sino también sociales y éticas, que son la aspectos básicos del desarrollo ágil.


Aunque las formas y aplicación de la enseñanza de estos temas varían de institución en institución, en la literatura prevalece la opinión de que la mejor forma de aprender desarrollo ágil es mediante el enfoque de aprender-haciendo (\emph{learn by doing}). Se encontraron ejemplos en donde se introduce al aprendizaje en prácticas de desarrollo ágil por medio de la implementación de algún trabajo final de graduación\footnote{\emph{Capstone project}}\cite{ding-yousef-yue}, proyectos/casos de estudio que se desarrollan durante un semestre\cite{steghoger-et-al}, desarrollando juegos\cite{scharlau}, por medio de laboratorios de robótica \cite{schroeder-et-al}, Wikis \cite{cubric} o bien como un tema paralelo al principal de un curso. Por ejemplo en \cite{haaranen-lehtinen}, se introduce a los estudiantes al versionamiento de código como herramienta de apoyo al desarrollo de una aplicación Web. De acuerdo al estudio en \cite{kropp-meier-2} el dominio de habilidades técnicas y prácticas de ingeniería representan la base para el entendimiento de la aplicación de otros temas sobre metodologías ágiles y para ir desarrollando una mentalidad dirigida al desarrollo de software de calidad.

Con respecto a la enseñanza de DevOps en universidades, se siguen un enfoques similares a los anteriormente mencionados para las prácticas de desarrollo ágil pero con un mayor énfasis en la introducción en temas de automatización \cite{henrik-b, bang-et-al, betz-et-al} de procesos de construcción, pruebas, entrega y computación en la nube. Un caso interesante es el que se expone en \cite{hickey-salas}, ``un campo de entrenamiento para emprendedores'', el cual es básicamente un curso de verano dedicado a ejercitar al máximo las habilidades de programación de los estudiantes y por medio de esto ejercitar también prácticas ágiles de desarrollo y de DevOps.





\subsection{Prácticas Ágiles de Desarrollo en Planes de Estudio de Referencia}
Desde la década de los sesenta, la \emph{American Computing Machinery} (ACM) junto con sociedades profesionales y científicas en computación se han dado a la tarea de propoponer y adaptar  recomendaciones en los planes de estudio debido al panorama cambiante de la tecnología\cite{acm-curriculum}.

En la guía curricular recomendada para los programas de bachillerato para la carrera de ciencias de la computación en \cite{acm-cs-curriculum} se incluyen temas tales como integración contínua, automatización, control de versiones, estándares de codificación, desarrollo orientado a pruebas, pruebas unitarias y pruebas de integración como parte del cuerpo de conocimiento de ingeniería de software sugerido para esta carrera. Para la guía curricular recomendada para los programas de ingeniería de software \cite{acm-se-curriculum} se incluyen temas de prácticas ágiles de desarrollo de software en las secciones de Validación y Verificacion, y en la sección de Procesos asociados al desarrollo de software. También se pudo constatar la prescencia de prácticas ágiles de desarrollo en cursos de referencia sobre  programación, ingeniería de requerimientos, proyecto finales, entre otros, que han desarrollado universidades en el extranjero. En \cite{acm-msis-curriculum} las prácticas ágiles de desarrollo aparecen como parte de las competencias del área de desarrollo y entrega de sistemas. 

Los autores de \cite{betz-et-al} junto con un grupo de trabajo formado por investigadores de universidades del estado de Minnesota en Estados Unidos, proponen un plan de estudios para carreras de sistemas de información y de tecnologías de la información que responde a las tendencias de desarrollo ágil y DevOps \cite{advance-it} de la actualidad. La propuesta que presentan fue la que se pudo identificar como la más radical hacia en la enseñanza y aprendizaje de prácticas ágiles de desarrollo.

\section{Enseñanza de Prácticas Ágiles en Costa Rica} \label{sec:costa-rica}
La evidencia recolectada apunta a que los primeras iniciativas dirigidas a la introducción y enseñanza de prácticas ágiles en las universidades de Costa Rica se dio a inicios del 2000. El siguiente fragmento tomado de \cite{cenfotec-2} reseña muy bien la situación de ese momento y da a conocer el porqué las universidades en Costa Rica empezaron a revisar sus planes de estudio para que fueran más acordes con la tendencias de la industria: \emph{``A fines de la década de 1990, la demanda de profesionales y técnicos en Informática tenía un crecimiento cercano al 90\% interanual Norteamérica, Europa Occidental y Japón. En Costa Rica, la industria costarricense de software crecía, en personal, entre un 40\% y 60\% anualmente, gracias a su éxito en mercados internacionales. La sostenibilidad de esa industria radica no en su costo, sino en la calidad de sus productos, la cual depende fundamentalmente del talento humano y de los sistemas de trabajo (procesos) aplicados. El cambio en la tecnología de software y el aumento en la demanda de aplicaciones informáticas – particularmente para la Web y los dispositivos móviles – han exigido innovaciones en la forma de educar profesionales en desarrollo de software preparados para producir aplicaciones y componentes de software de calidad mundial. Además, desde el año 2003, Costa Rica participa como proveedor de servicios tecnológicos y empresariales habilitados por tecnologías digitales. Esto ha planteado la necesidad de preparar profesionales con perfiles distintos de los que se graduaban de las universidades costarricenses.''}

En sus inicios, la hoy Universidad Cenfotec \cite{cenfotec-1} presentó un plan de estudios para el programa de Especialista en Tecnología de Software, el cual se diferenciaba de los carreras universitarias convenciales al sacrificar la generalidad en aras de la especialidad y permitir al graduado incorporarse al mercado laboral para realizar desarrollo de software de alta calidad. La intención detrás de este programa era ejercitar de forma intensiva planificación, diseño y programación de software a partir de la ejecución de proyectos y del enfoque de aprender-haciendo.

La Universidad de Costa Rica ha reportado la introducción de \emph{Scrum} y programación extrema como parte de sus cursos en ingeniería de software. En \cite{salazar} se reporta el uso de estas prácticas junto con RUP. El profesor a cargo establece un proyecto que se va desarrollando durante todo el semestre. La Universidad Nacional reporta la ejecución de iniciativas para impartir cursos de programación bajo modelos pedagógicos orientados a proyectos y a resolución de problemas \cite{mora-et-al-1, mora-et-al-2}, sin embargo en este caso, la experiencia reportada no incluye prácticas ágiles de desarrollo sino más bien lo que se intenta es propiciar el aprendizaje colaborativo, la participación y la autonomía. Cenfotec también promueve este enfoque de aprendizaje colaborativo y orientado a resolución de problemas\cite{trejos-1, cenfotec-2} por medio proyectos integradores los cuales aparte de formar habilidates técnicas o fuertes, también pretenden formar en habilidades suaves como comunicación, creatividad, valores éticos, agilidad, educación continua, entre otros.

\subsection{Sobre la experiencia en la enseñanza de prácticas ágiles de desarrollo}

\subsection{Beneficios}
\begin{itemize}
    \item Uso de herramientas de software de actualidad y un aprendizaje más alineado con las tendencias de la industria\cite{cenfotec-2}.
    \item Representa para muchos estudiantes la primera experiencia en donde se codifica, prueba, integra y se pone en producción un sistema, recibiendo en cada paso retroalimentación inmediata\cite{kropp-meier-2}.
    \item En \cite{kropp-meier-1} se reporta una mejora de las habilidades de programación y calidad del código.
    \item De acuerdo a \cite{scharlau} los estudiantes sentían que se rompía la rutina y que se daba mayor paso a la espontaneidad y la innovación. Esto también propiciaba que los estudiantes tuvieran mayor compromiso en los proyectos en los que trabajaron.
    \item La actividad en cuanto a los emprendimientos llevados a cabo por los estudiantes aumenta considerablemente luego de ser introducidos a estos temas\cite{hickey-salas}.
    \item Repetir la misma metodología y herramientas en cada \emph{sprint} mejoró la productividad de desarrollo y esto revela una mejoría de la curva de aprendizaje. Conforme fueron avanzando el desarrollo de los \emph{sprints}, las horas requeridas de esfuerzo para desarrollar una funcionalidad fue disminuyendo, por lo tanto la productividad mejoró\cite{salazar}.
    \item La introducción de prácticas ágiles de desarrollo fortaleció el proceso de pruebas, asegurando mayor calidad y evitando el re-trabajo\cite{salazar}
    \item Las prácticas ágiles de desarrollo de software, al ser el aspecto más técnico en las metodologías ágiles resulta ser también el más atractivo para los estudiantes y en donde concentran más sus esfuerzos de aprendizaje\cite{steghoger-et-al}
    \item La mayoría de la evidencia señala que la aplicación de la enseñanza de prácticas ágiles se hace a través de enfoques orientados a resolución de problemas y proyectos. Los resultados obtenidos en \cite{mora-et-al-1} resaltan la importancia de estos enfoques porque ofrecen a los estudiantes mayores posibilidades de tomar decisiones cuando se trabaja con la solución de problemas, y en la forma de abordarlo y, como consecuencia, aumentar su motivación en el proceso de aprendizaje y de asumir mayor responsabilidad.
    \item Hazzan y Dubinsky \cite{hazzan-dubinsky} señalan otros beneficios asociados a la enseñanza de prácticas ágiles de desarrollo tales como:
    \begin{itemize}
        \item Su aplicabilidad en la industria
        \item Trabajo en equipo
        \item Apoya otros procesos de aprendizaje
        \item Lidia con aspectos humanos, promueve la diversidad, habilidades de gestión y normas éticas
        \item Promueve hábitos mentales
        \item Brinda una imagen detallada de ingeniería de software
        \item Las metodologías ágiles proporcionan un marco de trabajo que se puede enseñar
    \end{itemize}
\end{itemize}

\subsection{Retos} \label{sec:retos}
\begin{itemize}

    \item Las instituciones educativas son lentas en crear e innovar en los planes de estudio para preparar a la futura fuerza laborar y desarrollar programas para re-entrenar aquellos que se ya se encuentran trabajando \cite{cenfotec-2}.
    \item Existen muy pocos artículos publicados acerca de la enseñanza de DevOps\cite{henrik-b}.
    \item En \cite{steghoger-et-al} se señala que:
        \begin{itemize}
            \item Cuando se enseñan procesos de desarrollo ágil debe de haber retroalimentación rápida y directa por parte de los profesores. La retroalimentación permite la corrección temprana a prácticas que se aplican mal.
            \item La planificación de cursos en estos temas requiere de un gran esfuerzo de adquisición de conocimiento previo, maxime si los potenciales instructores han tenido poca exposición a los mismos.
            \item La enseñanza de metodologías ágiles necesita tener un elemento práctico, se necesita aplicar para que se pueda saber. Cursos en donde se exponen solamente aspectos teóricos resultan menos atractivos para los estudiantes. 
        \end{itemize} 
        
        \item Si la formación en tecnologías ágiles y DevOps están por fuera de los planes regulares y se brinda en la modalidad de curso de verano o electivo, esto puede llegar a necesitar de mucho esfuerzo por parte de unidad académica a cargo y reportar pocas ganancias\cite{hickey-salas}.
        \item No es conveniente que se aprendan y apliquen metodologías ágiles como único método de enseñanza en cursos de ingeniería de software, porque generalmente estos modelos de desarrollo promueven un proceso informal, con escasa documentación y con un mínimo de productos de trabajo de ingeniería de software que atentan con la calidad del software y otras prácticas recomendadas\cite{salazar}
        \item En \cite{mora-et-al-1} se recalca la importancia de contar con personal motivado y comprometido en adoptar nuevos cambios en los cursos de la carrera: \emph{``Además, aunque la mayoría de los docentes manifiestan estar abiertos a hacer cambios que mejoren el aprendizaje y motiven más a los estudiantes, es evidente que estos procesos implican más trabajo y un cambio de cultura de los actores involucrados. Además es claro que se requiere de una línea de gestión clara y de una estructura de gestión curricular flexible que permita modificar las estructuras de evaluación tradicionales''}.
        \item Kropp y Meier \cite{kropp-meier-1} opinan que los cursos sobre desarrollo ágil de software no pueden ser enseñados en cursos aislados de ingeniería de software. Un reto es la integración de prácticas de desarrollo ágil en otros cursos como programación o análisis y diseño orietado a objetos, algoritmos y estruturas de datos.
\end{itemize}