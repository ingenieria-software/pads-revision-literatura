\section{Descripción de Puntos de Control}

Para la ejecución de este propuesta se propone dos etapas: evaluación de las carreras acreditadas por SINAES primero y luego las carreras no acreditadas.

\subsection{Evaluación de las carreras acreditadas por SINAES}
Se encontraron 8 carreras acreditadas por SINAES. Al contar con esta acreditación las universidades se ha comprometido con el cumplimiento de los criterios de calidad del SINAES. En la primera etapa del estudio, se evaluarán a estas 8 carreras primero. Con esto lo que se busca es:
\begin{itemize}
    \item Evaluar el nivel de penetración de las prácticas ágiles de desarrollo en las carreras acreditadas por SINAES
    \item Obtener experiencia en la ejecución del estudio, lo cual permitiría afinar herramientas de recolección y análisis de datos
    \item Utilizar los resultados de estas universidades como la primer referencia para comparar con las universidades no acreditadas
\end{itemize}

Una vez que se haya concluido con la evaluación de estas 8 carreras, el equipo de investigación podrá hacer un alto en el camino y examinar la experiencia y resultados obtenido.

\subsection{Evaluación de carreras no acreditadas}
Se procederá con la evaluación de las carreras no acreditadas y se aprovechará la experiencia y el trabajo realizado del punto anterior. Para evaluar estas carreras no se propone un enfoque en particular, lo que hace que quede a jucio del grupo de investigación.


\section{Cronograma}

\subsection{Para la evaluación de carreras acreditadas por SINAES}
\begin{itemize}
    \item Recolección de datos (7 días)
    \item Observaciones preliminares (2 días) 
    \item Preparación de entrevistas y encuestas (2 días)
    \item Ejecución de entrevista y encuestas (7 días, +/- 2 entrevistas por día)
    \item Análisis e Interpretación de datos (7 días)
    \item Total: 25 días
\end{itemize}

\subsection{Para la evaluación de carreras no acreditadas por SINAES}
\begin{itemize}
    \item Recolección de datos (15 días)
    \item Observaciones preliminares (7 días) 
    \item Preparación de entrevistas y encuestas (2 días)
    \item Ejecución de entrevista y encuestas (30 días)
    \item Análisis e Interpretación de datos (15 días)
    \item Total: 69 días
\end{itemize}

Total General: 94 días. 

Inicialmente este cronograma está planeado de forma ``pesimista'' principalmente porque las tareas asociadas a entrevistas se suponen de forma presencial, lo cual implica traslados constantes. En el caso en que se puedan planear entrevistas de forma remota el calendario podría verse acortado.