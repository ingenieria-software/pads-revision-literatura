\section{Situación Actual y Esperada}
El desarrollo de software siguiendo metodologías ágiles goza de mucha aceptación dentro de la comunidad de profesionales. Sus orígenes datan de inicios de la década de los 2000 por lo que no se puede argumentar que es una nuevo paradigma, pero como se ha destacado en esta propuesta la adopción  de estos principios ha calado de forma lenta dentro de los planes de estudio de universidades internacionales y de Costa Rica.

Estudios de este tipo se han hecho principalmente en Estados Unidos y en Europa, los cuales han motivado que las universidades hayan empezado a ofrecer a los estudiantes alternativas para enseñar sobre desarrollo de software ágil y prácticas ágiles de desarrollo de software.

En Costa Rica, de acuerdo con los artículos y reportes recopilados se prevee que el grado de adopción de estas prácticas sea de medio a bajo. Informes de la industria como \cite{prosic, murillo-trejos} señalan un descontento por parte de los empleadores los cuales argumentan que el personal que contratan tienen carencias para incorporarse de forma efectiva al trabajo y les toma tiempo poder conceptualizar nuevas habiliidades. También, en una observación inicial que se hizo a cursos relacionados con desarrollo de software en los planes de estudio de carreras de computación, informática y sistemas de información de 3 universidades se pudo notar que no se dedica mucho tiempo al desarrollo de estas habilidades paralelas a la programación.

Dentro del análisis inicial, la Universidad Cenfotec se destaca en exponer a sus estudiantes a muchas horas en el estudio de casos de estudio, problemas y al desarrollo de habilidades relacionadas en desarrollo de software. Este se considera un elemento diferenciador con respecto a otras carreras de Ingeniería de Software que se ofrecen en el país las cuales siguen esquemas tradicionales. Dejando por fuera al Cenfotec, no se pudo encontrar una universidad que brindara un curso el cual llevara a los estudiantes desde la creación y gestión del proyecto de software hasta su puesta en producción, pasando en el camino por temas de buenas prácticas, pruebas e integración

Aunque se espera que la penetración de las prácticas ágiles de desarrollo de software no sea alta, debido a los énfasis y enfoques propios de cada carrera, también sería de agrado saber que en la universidades se reconoce el valor de su enseñanza, como medio para formar profesionales con mejores armas para enfrentar sus retos profesionales futuros, y que se tiene planes para llevar acciones al respecto.


\section{Propuesta de la Solución}
Basado en los casos de estudio de universidades extranjeras se proponen las siguientes soluciones para brindar enseñanza de prácticas ágiles de desarrollo de software en los planes de estudio de carreras de computación, informática y sistemas de información en Costa Rica:

\subsection{Creación de un curso de laboratorio/taller/práctica en desarrollo de software}
Este curso pretende sintetizar el conocimiento adquirido en desarrollo de software por los estudiantes durante la carrera con el fin de aplicarlo en la implementación de una aplicación. En la implementación se pretende cubrir los principales temas relacionados con prácticas ágiles de desarrollo y además se le puede dar enseñanza de primera mano al estudiante de cómo se configura un proyecto de software desde su inicio y cuáles son las estrategias adecuadas para ponerlo en producción.

Muchos cursos de programación están enfocados a resolver problemas muy específicos y en ocasiones no hay espacio para desarrollar habilidades paralelas en los estudiantes. Este curso serviría como complemento para dar a conocer a los estudiantes otras posibilidades que pueden explotar por medio del software.

Los temas de este curso serían:
\begin{enumerate}
    \item Construcción y administración de proyectos de software
    \begin{itemize}
        \item[-] Conformación de equipos de trabajo
        \item[-] Selección del tipo de aplicación a desarrollar y sus requerimientos
        \item[-] Selección de lenguaje de programación - herramientas de construcción
        \item[-] Configuración de repositorio de código
    \end{itemize}
    \item{Herramientas de desarrollo}
    \begin{itemize}
        \item[-] Buenas prácticas de codificación / refactorización
        \item[-] Herramientas propias del lenguaje de programación seleccionado
        \item[-] Integración continua 
    \end{itemize}
    \item Introducción a pruebas
    \begin{itemize}
        \item[-] Pruebas unitarias / Integración
        \item[-] Desarrollo orientado a pruebas (TDD)
        \item[-] Pruebas automatizadas 
    \end{itemize}
    \item Puesta en producción
    \begin{itemize}
        \item[-] Configuración de infraestructura
        \item[-] Introducción a ambientes de producción en la nube
        \item[-] Introducción a enfoques de entrega continua
    \end{itemize}
    \item Otros temas
    \begin{itemize}
        \item[-] Escalabilidad
        \item[-] Rendimiento 
        \item[-] Seguridad
    \end{itemize}
\end{enumerate}


Los temas anteriores se irán desarrollando y evaluando por etapas y al final se debería contar con una aplicación funcional que podrá ser accesada por medio de Internet o alguna otra plataforma como lo podría ser una aplicación móvil.


\subsection{Integración en cursos existentes}
En esta modalidad se propone que el personal de la universidad evalue cuáles prácticas de desarrollo se acoplan de mejor manera en los temas de cursos ya existentes. Por ejemplo, se podrían utilizar cursos introductorios a la programación para presentar a los estudiantes a temas de configuración y gestión de proyectos de software, los cuales sin duda alguna les será de utilidad para cursos futuros.

Luego, en cursos de temas avanzados, se puede introducir a temas de buenas prácticas de programación e integración. En cursos sobre aseguramiento de la calidad o ingenieria de requerimientos, los temas de pruebas podrían ser incluidos. Se recomienda que, de adoptarse este esquema, los temas que se incluyan sobre prácticas ágiles de desarrollo de software, no se traten de forma tangencial sino que se pueda ejercitar su enseñanza por medio de actividades que motiven a los estudiantes a desarrollar algún caso de estudio.


\subsection{Talleres/Charlas impartidos por terceros}
Las universidades pueden planear la ejecución de talleres y charlas sobre estos temas con empresas y profesionales. Esta es también una opción interesante porque le permite a los estudiantes ponerse en contacto con las tendencias más actuales de la industria y evaluar posibles temas en los que quisieran investigar o especializarse en un futuro.