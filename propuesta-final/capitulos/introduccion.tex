En las últimas décadas se ha hecho un esfuerzo significativo en identificar buenas prácticas, modelos y métodos que conduzcan a desarrollar software de forma más eficiente. Los desarrolladores de software tienden a clasificar las metodologías de desarrollo en dos categorías\cite{li-jian-armin-eberlein}:
\begin{enumerate}
    \item Metodologías de desarrollo clásicas: requieren definición de requerimientos por adelantado, documentación y planes detallados. Dos ejemplos relevantes de son el modelo de cascada y espiral.
    \item Metodologías ágiles: a menudo se llama ``livianas''. Esta categoría incluye \emph{eXtreme Programming} (XP) y \emph{Scrum}.
\end{enumerate}

El movimiento de desarrollo ágil de software nace como una alternativa a los métodos tradicionales con los que se hace software, en los cuales los ciclos de desarrollo y entrega tienden a ser muy  prolongados. El desarrollo ágil de software se define en el Manifiesto Ágil \cite{agile-manifesto} como un conjunto de doce principios. 

Reportes sobre el uso de metodologías ágiles en desarrollo de software muestran un sostenido crecimiento a través de los años\cite{version-one} así como la literatura de investigación con respecto al impacto de estos enfoques en varios aspectos del ciclo de vida de desarrollo como lo puede ser la calidad, la entrega, gestión de requerimientos, entre otros. 

Otra práctica que ha resultado tener gran impacto en el sector del desarrollo de software en los años muy reciente es DevOps. Esta práctica promueve retomar viejos paradigmas de trabajo en donde el ingeniero/programador/investigador, estaba a cargo de todo el proceso de desarrollo: desde el diseño y la programación hasta la generación de pruebas, artefactos de software e instalación final. DevOps ha ganado mucho terreno gracias a la computación en la nube en donde en lugar de estar a cargo de recursos físicos, se está a cargo de recursos virtuales por lo que los roles del programador que solo programa, el ingeniero de pruebas que solo prueba software y finalmente el administrador de servidores que solamente se dedica a configurar hardware, se motivan a ser re-pensados. En ambientes como el de la computación en la nube, la infraestrutura está en el código lo que hace que al trabajar con este tipo de aplicaciones se tiene que estar consciente de muchas más cosas. De esta forma se puede decir que DevOps es la combinación entre desarrollo y operaciones. 

A pesar de ser un enfoque de desarrollo aún más reciente que las metodologías ágiles, DevOps ha sido influenciado por muchos de los principios de estas, principalmente del \emph{eXtreme Programming}. DevOps es visto por muchos como una extensión natural de las prácticas ágiles\cite{henrik-b} y actualmente se puede encontrar mucha literatura disponible sobre la aplicación de metodologías ágiles junto con DevOps.

Sin embargo la rápida penetración y auge de estas prácticas viene con un precio a pagar. En los primeros años de adopción de las metodologías ágiles era muy difícil encontrar personal con las habilidades necesarias, esto forzaba tanto a empleados como a empleadores a buscar formas alternas de formación. Por otro lado, los graduados de carreras de tecnologías de información del momento tampoco contaban con tales conocimiento debido a que los planes de estudio abarcaban poco o nada de estas prácticas. En los círculos académicos no había mucho interés en desarrollar estos temas debido a que se consideraba que desarrollo ágil no tenía bases teóricas sino que más bien ha sido desarrollada a partir de la práctica y la experiencia\cite{hazzan-dubinsky}. Esto ha venido cambiando de forma substancial y ahora temas sobre metodologías ágiles se han incorporado a los planes de estudio de carreras de tecnologías de la información de formas diversas. Aún se acusan muchos retos asociados a la enseñanza de estas prácticas, tal y como se indica en la Sección \ref{sec:retos}.

Con el paso de los años se ha constatado que estas prácticas han dejado de ser una moda o una tendencia del momento y que paralelo a fomentar mejoras en aspectos propios de ingeniería, también promueven el trabajo en equipo y colaborativo, habilidades de gestión, normas éticas y emprendedurismo \cite{hazzan-dubinsky, hickey-salas}. 

Para apoyar la mejora de habilidades técnicas y la inserción laboral, varios enfoques de enseñanza de ingeniería de software utilizando prácticas ágiles han sido reportados, \cite{ding-yousef-yue, steghoger-et-al, scharlau, schroeder-et-al, cubric, haaranen-lehtinen, kropp-meier-2}. Estas iniciativas persiguen exponer a los estudiantes a ambientes de trabajo similares a los que se encontraran en su práctica profesional y a la mejora de sus habilidades en desarrollo de software. Cobra aún mayor importancia la exposición de estos temas en la universidad si se toma en cuenta el gran auge de la computación en la nube. Baker \cite{advance-it} señala que \emph{``Los estudiantes de hoy en día puede ser que nunca se enfrenten a un servidor físico, un switch de red o un dispositivo de almacenamiento durante su carrera profesional futura pero aún así van a estar construyendo más infraestructura de IT que todos sus predecesores combinados. Se necesita enseñar a los estudiantes en cómo crear infraestructuas usando códido y cómo trabajar juntos de forma colaborativa para gestionar aplicaciones distribuidas complejas''}. 


En Costa Rica, la industria de desarrollo de software ha tenido un gran crecimiento en los últimos 20 años \cite{cenfotec-2, prosic}. A pesar del crecimiento, las organizaciones reportan deficiencias en la formación de la fuerza laboral y en los planes de estudio que, según señalan, no están actualizados con las necesidades del sector \cite{prosic}. Con respecto a la enseñanza de prácticas ágiles, las iniciativas en \cite{trejos-1, salazar, mora-et-al-1, mora-et-al-2} dan a conocer enfoques en cómo se están introduciendo a los estudiantes en estos temas. Se destacan los enfoques de enseñanza orientados a resolución de problemas y proyectos.