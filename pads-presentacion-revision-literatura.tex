\documentclass[14pt]{beamer}

    \usepackage[utf8]{inputenc}
    \usepackage{graphicx,hyperref,url}
    \usepackage{tikz}
    \usepackage{wrapfig}
    \usepackage{subcaption}
    \usepackage{epigraph}
    \usepackage{framed}
    \usepackage{palatino}
    \graphicspath{ {images/} }
    

% Titulo
    \title[Revisión de Literatura]
    {
        {\fontfamily{ppl}\selectfont{\textbf{Adopción de prácticas ágiles de desarrollo de software en los planes de estudio de universidades de Costa Rica: revisión de la literatura}}}
    } % define title
    \author[Carlos Martín Flores González] % (optional, use only with lots of authors)
    {Carlos Martín Flores González\\
        {\small \url{martin.flores@ieee.org}}
    }
    \date{\scriptsize{21 de setiembre 2017}}
    \institute[Instituto Tecnológico de Costa Rica]
    {
      Escuela de Ingeniería en Computación\\
      Instituto Tecnológico de Costa Rica
      \\
      Ingeniería de Software\\
      Profesor: Rodrigo Bogarin
    }
    
% Personalizacion del documento
    \usefonttheme{professionalfonts}
%    \usetheme{Berlin}
%    \usecolortheme{dolphin}
%    \setbeamercovered{transparent}
%    \beamertemplatenavigationsymbolsempty
    
%    \pgfdeclareimage{tec-logo}{images/tec-logo.jpg}
%    \usebackgroundtemplate{
%        \tikz[overlay,remember picture] \node[opacity=0.0, at=(current page.center)] {
%        \includegraphics[height=4.88cm,width=5cm]{tec-logo}};      
%    }
\def\Put(#1)#2{\leavevmode\makebox(0,0){\put(#1){#2}}}
    
    
\begin{document}
    {
%    \usebackgroundtemplate{}
    \begin{frame}[plain]
      \titlepage
    \end{frame}
    }
   
% =========================================
%
% =========================================

    

    \begin{frame}{Agenda}

       \begin{enumerate}
           \item
           Definiendo el problema de investigación
           \item
           Subproblemas
           \item
           Refinamiento de problemas
           \item
           Propuesta de investigación
           \item
           Conclusiones

       \end{enumerate}

    \end{frame}

    \section*{El Libro}

    
    \section*{Definiendo el problema de investigación}
    
    \subsection*{Proyectos de investigación}
   
   
   \section{Subproblemas}
   
   \subsection*{Dividiendo el problema de investigación en Subproblemas}
 

    \section{Refinamiento de problemas}
    
    \subsection{Cada problema necesita refinamiento}
     

    
%%%%%%%%%%%%%%%%%%%%%%%%%%%%%%%%%%%%%%%%%%%%%%%%%
    
        \begin{frame}{Conclusiones}
        
            \begin{itemize}[<+-|alert@+>]
            
                \item
                Recordar que el proyecto va a tomar tiempo
                
                \item
                Revisión detallada de la literatura disponible
                
                \item
                Intentar ver el problema de diferentes perspectivas
                
                \item
                Refinar el problema y sus alcances constantemente
                
                \item
                Usar todos los recursos disponibles
                
                \item
                Discutir el problema con otros
               
            \end{itemize}

            
        \end{frame}

        
       \begin{frame}[plain]
        \begin{center}
            \Large{\textbf{¡Muchas Gracias!}}
        
        
            \epigraph{``Todo lo saqué de mi mente''}
            {--- Estudiante de colegio, a profesor de matemática}
        \end{center}
    \end{frame}
    
\end{document}