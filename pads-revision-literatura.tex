
%% bare_jrnl_transmag.tex
%% V1.4a
%% 2014/09/17
%% by Michael Shell
%% see http://www.michaelshell.org/
%% for current contact information.
%%
%% This is a skeleton file demonstrating the use of IEEEtran.cls
%% (requires IEEEtran.cls version 1.8a or later) with an IEEE 
%% Transactions on Magnetics journal paper.
%%
%% Support sites:
%% http://www.michaelshell.org/tex/ieeetran/
%% http://www.ctan.org/tex-archive/macros/latex/contrib/IEEEtran/
%% and
%% http://www.ieee.org/

%%*************************************************************************
%% Legal Notice:
%% This code is offered as-is without any warranty either expressed or
%% implied; without even the implied warranty of MERCHANTABILITY or
%% FITNESS FOR A PARTICULAR PURPOSE! 
%% User assumes all risk.
%% In no event shall IEEE or any contributor to this code be liable for
%% any damages or losses, including, but not limited to, incidental,
%% consequential, or any other damages, resulting from the use or misuse
%% of any information contained here.
%%
%% All comments are the opinions of their respective authors and are not
%% necessarily endorsed by the IEEE.
%%
%% This work is distributed under the LaTeX Project Public License (LPPL)
%% ( http://www.latex-project.org/ ) version 1.3, and may be freely used,
%% distributed and modified. A copy of the LPPL, version 1.3, is included
%% in the base LaTeX documentation of all distributions of LaTeX released
%% 2003/12/01 or later.
%% Retain all contribution notices and credits.
%% ** Modified files should be clearly indicated as such, including  **
%% ** renaming them and changing author support contact information. **
%%
%% File list of work: IEEEtran.cls, IEEEtran_HOWTO.pdf, bare_adv.tex,
%%                    bare_conf.tex, bare_jrnl.tex, bare_conf_compsoc.tex,
%%                    bare_jrnl_compsoc.tex, bare_jrnl_transmag.tex
%%*************************************************************************


% *** Authors should verify (and, if needed, correct) their LaTeX system  ***
% *** with the testflow diagnostic prior to trusting their LaTeX platform ***
% *** with production work. IEEE's font choices and paper sizes can       ***
% *** trigger bugs that do not appear when using other class files.       ***                          ***
% The testflow support page is at:
% http://www.michaelshell.org/tex/testflow/



\documentclass[journal]{IEEEtran}
%
% If IEEEtran.cls has not been installed into the LaTeX system files,
% manually specify the path to it like:
% \documentclass[journal]{../sty/IEEEtran}





% Some very useful LaTeX packages include:
% (uncomment the ones you want to load)


% *** MISC UTILITY PACKAGES ***
%
%\usepackage{ifpdf}
% Heiko Oberdiek's ifpdf.sty is very useful if you need conditional
% compilation based on whether the output is pdf or dvi.
% usage:
% \ifpdf
%   % pdf code
% \else
%   % dvi code
% \fi
% The latest version of ifpdf.sty can be obtained from:
% http://www.ctan.org/tex-archive/macros/latex/contrib/oberdiek/
% Also, note that IEEEtran.cls V1.7 and later provides a builtin
% \ifCLASSINFOpdf conditional that works the same way.
% When switching from latex to pdflatex and vice-versa, the compiler may
% have to be run twice to clear warning/error messages.


\usepackage[utf8]{inputenc}
\usepackage[T1]{fontenc}
\usepackage[spanish]{babel}
\usepackage{hyperref}
\usepackage{graphicx}
\usepackage{booktabs}
\usepackage{cite}
\graphicspath{ {images/} }
\hypersetup{
    colorlinks=true,
    linkcolor=blue,
    filecolor=magenta,      
    urlcolor=cyan
}




% *** CITATION PACKAGES ***
%
%\usepackage{cite}
% cite.sty was written by Donald Arseneau
% V1.6 and later of IEEEtran pre-defines the format of the cite.sty package
% \cite{} output to follow that of IEEE. Loading the cite package will
% result in citation numbers being automatically sorted and properly
% "compressed/ranged". e.g., [1], [9], [2], [7], [5], [6] without using
% cite.sty will become [1], [2], [5]--[7], [9] using cite.sty. cite.sty's
% \cite will automatically add leading space, if needed. Use cite.sty's
% noadjust option (cite.sty V3.8 and later) if you want to turn this off
% such as if a citation ever needs to be enclosed in parenthesis.
% cite.sty is already installed on most LaTeX systems. Be sure and use
% version 5.0 (2009-03-20) and later if using hyperref.sty.
% The latest version can be obtained at:
% http://www.ctan.org/tex-archive/macros/latex/contrib/cite/
% The documentation is contained in the cite.sty file itself.






% *** GRAPHICS RELATED PACKAGES ***
%
\ifCLASSINFOpdf
  % \usepackage[pdftex]{graphicx}
  % declare the path(s) where your graphic files are
  % \graphicspath{{../pdf/}{../jpeg/}}
  % and their extensions so you won't have to specify these with
  % every instance of \includegraphics
  % \DeclareGraphicsExtensions{.pdf,.jpeg,.png}
\else
  % or other class option (dvipsone, dvipdf, if not using dvips). graphicx
  % will default to the driver specified in the system graphics.cfg if no
  % driver is specified.
  % \usepackage[dvips]{graphicx}
  % declare the path(s) where your graphic files are
  % \graphicspath{{../eps/}}
  % and their extensions so you won't have to specify these with
  % every instance of \includegraphics
  % \DeclareGraphicsExtensions{.eps}
\fi
% graphicx was written by David Carlisle and Sebastian Rahtz. It is
% required if you want graphics, photos, etc. graphicx.sty is already
% installed on most LaTeX systems. The latest version and documentation
% can be obtained at: 
% http://www.ctan.org/tex-archive/macros/latex/required/graphics/
% Another good source of documentation is "Using Imported Graphics in
% LaTeX2e" by Keith Reckdahl which can be found at:
% http://www.ctan.org/tex-archive/info/epslatex/
%
% latex, and pdflatex in dvi mode, support graphics in encapsulated
% postscript (.eps) format. pdflatex in pdf mode supports graphics
% in .pdf, .jpeg, .png and .mps (metapost) formats. Users should ensure
% that all non-photo figures use a vector format (.eps, .pdf, .mps) and
% not a bitmapped formats (.jpeg, .png). IEEE frowns on bitmapped formats
% which can result in "jaggedy"/blurry rendering of lines and letters as
% well as large increases in file sizes.
%
% You can find documentation about the pdfTeX application at:
% http://www.tug.org/applications/pdftex




% *** MATH PACKAGES ***
%
%\usepackage[cmex10]{amsmath}
% A popular package from the American Mathematical Society that provides
% many useful and powerful commands for dealing with mathematics. If using
% it, be sure to load this package with the cmex10 option to ensure that
% only type 1 fonts will utilized at all point sizes. Without this option,
% it is possible that some math symbols, particularly those within
% footnotes, will be rendered in bitmap form which will result in a
% document that can not be IEEE Xplore compliant!
%
% Also, note that the amsmath package sets \interdisplaylinepenalty to 10000
% thus preventing page breaks from occurring within multiline equations. Use:
%\interdisplaylinepenalty=2500
% after loading amsmath to restore such page breaks as IEEEtran.cls normally
% does. amsmath.sty is already installed on most LaTeX systems. The latest
% version and documentation can be obtained at:
% http://www.ctan.org/tex-archive/macros/latex/required/amslatex/math/





% *** SPECIALIZED LIST PACKAGES ***
%
%\usepackage{algorithmic}
% algorithmic.sty was written by Peter Williams and Rogerio Brito.
% This package provides an algorithmic environment fo describing algorithms.
% You can use the algorithmic environment in-text or within a figure
% environment to provide for a floating algorithm. Do NOT use the algorithm
% floating environment provided by algorithm.sty (by the same authors) or
% algorithm2e.sty (by Christophe Fiorio) as IEEE does not use dedicated
% algorithm float types and packages that provide these will not provide
% correct IEEE style captions. The latest version and documentation of
% algorithmic.sty can be obtained at:
% http://www.ctan.org/tex-archive/macros/latex/contrib/algorithms/
% There is also a support site at:
% http://algorithms.berlios.de/index.html
% Also of interest may be the (relatively newer and more customizable)
% algorithmicx.sty package by Szasz Janos:
% http://www.ctan.org/tex-archive/macros/latex/contrib/algorithmicx/




% *** ALIGNMENT PACKAGES ***
%
%\usepackage{array}
% Frank Mittelbach's and David Carlisle's array.sty patches and improves
% the standard LaTeX2e array and tabular environments to provide better
% appearance and additional user controls. As the default LaTeX2e table
% generation code is lacking to the point of almost being broken with
% respect to the quality of the end results, all users are strongly
% advised to use an enhanced (at the very least that provided by array.sty)
% set of table tools. array.sty is already installed on most systems. The
% latest version and documentation can be obtained at:
% http://www.ctan.org/tex-archive/macros/latex/required/tools/


% IEEEtran contains the IEEEeqnarray family of commands that can be used to
% generate multiline equations as well as matrices, tables, etc., of high
% quality.




% *** SUBFIGURE PACKAGES ***
%\ifCLASSOPTIONcompsoc
%  \usepackage[caption=false,font=normalsize,labelfont=sf,textfont=sf]{subfig}
%\else
%  \usepackage[caption=false,font=footnotesize]{subfig}
%\fi
% subfig.sty, written by Steven Douglas Cochran, is the modern replacement
% for subfigure.sty, the latter of which is no longer maintained and is
% incompatible with some LaTeX packages including fixltx2e. However,
% subfig.sty requires and automatically loads Axel Sommerfeldt's caption.sty
% which will override IEEEtran.cls' handling of captions and this will result
% in non-IEEE style figure/table captions. To prevent this problem, be sure
% and invoke subfig.sty's "caption=false" package option (available since
% subfig.sty version 1.3, 2005/06/28) as this is will preserve IEEEtran.cls
% handling of captions.
% Note that the Computer Society format requires a larger sans serif font
% than the serif footnote size font used in traditional IEEE formatting
% and thus the need to invoke different subfig.sty package options depending
% on whether compsoc mode has been enabled.
%
% The latest version and documentation of subfig.sty can be obtained at:
% http://www.ctan.org/tex-archive/macros/latex/contrib/subfig/



% *** FLOAT PACKAGES ***
%
%\usepackage{fixltx2e}
% fixltx2e, the successor to the earlier fix2col.sty, was written by
% Frank Mittelbach and David Carlisle. This package corrects a few problems
% in the LaTeX2e kernel, the most notable of which is that in current
% LaTeX2e releases, the ordering of single and double column floats is not
% guaranteed to be preserved. Thus, an unpatched LaTeX2e can allow a
% single column figure to be placed prior to an earlier double column
% figure. The latest version and documentation can be found at:
% http://www.ctan.org/tex-archive/macros/latex/base/


%\usepackage{stfloats}
% stfloats.sty was written by Sigitas Tolusis. This package gives LaTeX2e
% the ability to do double column floats at the bottom of the page as well
% as the top. (e.g., "\begin{figure*}[!b]" is not normally possible in
% LaTeX2e). It also provides a command:
%\fnbelowfloat
% to enable the placement of footnotes below bottom floats (the standard
% LaTeX2e kernel puts them above bottom floats). This is an invasive package
% which rewrites many portions of the LaTeX2e float routines. It may not work
% with other packages that modify the LaTeX2e float routines. The latest
% version and documentation can be obtained at:
% http://www.ctan.org/tex-archive/macros/latex/contrib/sttools/
% Do not use the stfloats baselinefloat ability as IEEE does not allow
% \baselineskip to stretch. Authors submitting work to the IEEE should note
% that IEEE rarely uses double column equations and that authors should try
% to avoid such use. Do not be tempted to use the cuted.sty or midfloat.sty
% packages (also by Sigitas Tolusis) as IEEE does not format its papers in
% such ways.
% Do not attempt to use stfloats with fixltx2e as they are incompatible.
% Instead, use Morten Hogholm'a dblfloatfix which combines the features
% of both fixltx2e and stfloats:
%
% \usepackage{dblfloatfix}
% The latest version can be found at:
% http://www.ctan.org/tex-archive/macros/latex/contrib/dblfloatfix/




%\ifCLASSOPTIONcaptionsoff
%  \usepackage[nomarkers]{endfloat}
% \let\MYoriglatexcaption\caption
% \renewcommand{\caption}[2][\relax]{\MYoriglatexcaption[#2]{#2}}
%\fi
% endfloat.sty was written by James Darrell McCauley, Jeff Goldberg and 
% Axel Sommerfeldt. This package may be useful when used in conjunction with 
% IEEEtran.cls'  captionsoff option. Some IEEE journals/societies require that
% submissions have lists of figures/tables at the end of the paper and that
% figures/tables without any captions are placed on a page by themselves at
% the end of the document. If needed, the draftcls IEEEtran class option or
% \CLASSINPUTbaselinestretch interface can be used to increase the line
% spacing as well. Be sure and use the nomarkers option of endfloat to
% prevent endfloat from "marking" where the figures would have been placed
% in the text. The two hack lines of code above are a slight modification of
% that suggested by in the endfloat docs (section 8.4.1) to ensure that
% the full captions always appear in the list of figures/tables - even if
% the user used the short optional argument of \caption[]{}.
% IEEE papers do not typically make use of \caption[]'s optional argument,
% so this should not be an issue. A similar trick can be used to disable
% captions of packages such as subfig.sty that lack options to turn off
% the subcaptions:
% For subfig.sty:
% \let\MYorigsubfloat\subfloat
% \renewcommand{\subfloat}[2][\relax]{\MYorigsubfloat[]{#2}}
% However, the above trick will not work if both optional arguments of
% the \subfloat command are used. Furthermore, there needs to be a
% description of each subfigure *somewhere* and endfloat does not add
% subfigure captions to its list of figures. Thus, the best approach is to
% avoid the use of subfigure captions (many IEEE journals avoid them anyway)
% and instead reference/explain all the subfigures within the main caption.
% The latest version of endfloat.sty and its documentation can obtained at:
% http://www.ctan.org/tex-archive/macros/latex/contrib/endfloat/
%
% The IEEEtran \ifCLASSOPTIONcaptionsoff conditional can also be used
% later in the document, say, to conditionally put the References on a 
% page by themselves.




% *** PDF, URL AND HYPERLINK PACKAGES ***
%
%\usepackage{url}
% url.sty was written by Donald Arseneau. It provides better support for
% handling and breaking URLs. url.sty is already installed on most LaTeX
% systems. The latest version and documentation can be obtained at:
% http://www.ctan.org/tex-archive/macros/latex/contrib/url/
% Basically, \url{my_url_here}.




% *** Do not adjust lengths that control margins, column widths, etc. ***
% *** Do not use packages that alter fonts (such as pslatex).         ***
% There should be no need to do such things with IEEEtran.cls V1.6 and later.
% (Unless specifically asked to do so by the journal or conference you plan
% to submit to, of course. )


% correct bad hyphenation here
\hyphenation{op-tical net-works semi-conduc-tor}


\begin{document}
%
% paper title
% Titles are generally capitalized except for words such as a, an, and, as,
% at, but, by, for, in, nor, of, on, or, the, to and up, which are usually
% not capitalized unless they are the first or last word of the title.
% Linebreaks \\ can be used within to get better formatting as desired.
% Do not put math or special symbols in the title.
\title{Adopción de prácticas ágiles de desarrollo de software en los planes de estudio de universidades de Costa Rica: revisión de la literatura}



% author names and affiliations
% transmag papers use the long conference author name format.

\author{\IEEEauthorblockN{Carlos Martín Flores González, 
\IEEEauthorblockA{
Escuela de Ingeniería en Computación\\
Instituto Tecnológico de Costa Rica\\
Cartago, Costa Rica\\
\emph{Email:} \texttt{martin.flores@ieee.org}}
}% <-this % stops an unwanted space
\thanks{Este documento fue realizado durante el curso de Ingeniería de Software, impartido por el profesor Rodrigo Bogarín. Programa de Maestría en Computación, Instituto Tecnológico de Costa Rica. Segundo Semestre, 2017.}
\thanks{Recibido el 21 de setiembre del 2017.}}



% The paper headers
\markboth{Ingeniería de Software. Setiembre~2017}%
{Shell \MakeLowercase{\textit{et al.}}: Bare Demo of IEEEtran.cls for Journals}
% The only time the second header will appear is for the odd numbered pages
% after the title page when using the twoside option.
% 
% *** Note that you probably will NOT want to include the author's ***
% *** name in the headers of peer review papers.                   ***
% You can use \ifCLASSOPTIONpeerreview for conditional compilation here if
% you desire.




% If you want to put a publisher's ID mark on the page you can do it like
% this:
%\IEEEpubid{0000--0000/00\$00.00~\copyright~2014 IEEE}
% Remember, if you use this you must call \IEEEpubidadjcol in the second
% column for its text to clear the IEEEpubid mark.



% use for special paper notices
%\IEEEspecialpapernotice{(Invited Paper)}


% for Transactions on Magnetics papers, we must declare the abstract and
% index terms PRIOR to the title within the \IEEEtitleabstractindextext
% IEEEtran command as these need to go into the title area created by
% \maketitle.
% As a general rule, do not put math, special symbols or citations
% in the abstract or keywords.
\IEEEtitleabstractindextext{%
\begin{abstract}
A pesar de la popularidad de las metodologías ágiles de desarrollo de software en la industria, expresada en varias encuestas y estudios recientes, muchas organizaciones siguen reportando en estas mismas encuestas que se cuenta con poco personal capacitado con las habilidades necesarias y que los planes de estudio de carreras de tecnología de información muestran rezagos en estos temas. Con el fin de brindar una formación que no solamente incluya la enseñanza de prácticas ágiles sino también el desarrollo de otras actitudes como el trabajo en equipo y el empredurismo, las universidades han venido introduciendo paulativamente estas prácticas a sus planes de estudio principalmente por medio de enfoques de aprender-haciendo. Universidades en Costa Rica también han reportado la inclusión de estos temas en sus planes de estudio como respuesta a las tendencias mundiales y para apoyar la creciente industria de desarrollo de software del país. Este artículo presenta una revisión de literatura en sobre la aplicación de prácticas ágiles de desarrollo de software en universidades extranjeras y de Costa Rica.
\end{abstract}

% Note that keywords are not normally used for peerreview papers.
\renewcommand\IEEEkeywordsname{Palabras Clave}
\begin{IEEEkeywords}
agile software development, agile teaching, agile engieering practices, software engineering education
\end{IEEEkeywords}}



% make the title area
\maketitle


% To allow for easy dual compilation without having to reenter the
% abstract/keywords data, the \IEEEtitleabstractindextext text will
% not be used in maketitle, but will appear (i.e., to be "transported")
% here as \IEEEdisplaynontitleabstractindextext when the compsoc 
% or transmag modes are not selected <OR> if conference mode is selected 
% - because all conference papers position the abstract like regular
% papers do.
\IEEEdisplaynontitleabstractindextext
% \IEEEdisplaynontitleabstractindextext has no effect when using
% compsoc or transmag under a non-conference mode.







% For peer review papers, you can put extra information on the cover
% page as needed:
% \ifCLASSOPTIONpeerreview
% \begin{center} \bfseries EDICS Category: 3-BBND \end{center}
% \fi
%
% For peerreview papers, this IEEEtran command inserts a page break and
% creates the second title. It will be ignored for other modes.
\IEEEpeerreviewmaketitle



\section{Introducción}
% The very first letter is a 2 line initial drop letter followed
% by the rest of the first word in caps.
% 
% form to use if the first word consists of a single letter:
% \IEEEPARstart{A}{demo} file is ....
% 
% form to use if you need the single drop letter followed by
% normal text (unknown if ever used by IEEE):
% \IEEEPARstart{A}{}demo file is ....
% 
% Some journals put the first two words in caps:
% \IEEEPARstart{T}{his demo} file is ....
% 
% Here we have the typical use of a "T" for an initial drop letter
% and "HIS" in caps to complete the first word.
\IEEEPARstart{E}{n} las últimas décadas se ha hecho un esfuerzo significativo en identificar buenas prácticas, modelos y métodos que conduzcan a desarrollar software de forma más eficiente. Los desarrolladores de software tienden a clasificar las metodologías de desarrollo en dos categorías\cite{li-jian-armin-eberlein}:
\begin{enumerate}
    \item Metodologías de desarrollo clásicas: requieren definición de requerimientos por adelantado, documentación y planes detallados. Dos ejemplos relevantes de son el modelo de cascada y espiral.
    \item Metodologías ágiles: a menudo se llama ``livianas''. Esta categoría incluye \emph{eXtreme Programming} (XP) y \emph{Scrum}.
\end{enumerate}

El movimiento de desarrollo ágil de software nace como una alternativa a los métodos tradicionales con los que se hace software, en los cuales los ciclos de desarrollo y entrega tienden a ser muy  prolongados. El desarrollo ágil de software se define en el Manifiesto Ágil \cite{agile-manifesto} como un conjunto de doce principios. 

Reportes sobre el uso de metodologías ágiles en desarrollo de software muestran un sostenido crecimiento a través de los años\cite{version-one} así como la literatura de investigación con respecto al impacto de estos enfoques en varios aspectos del ciclo de vida de desarrollo como lo puede ser la calidad, la entrega, gestión de requerimientos, entre otros. 

Otra práctica que ha resultado tener gran impacto en el sector del desarrollo de software en los años muy reciente es DevOps. Esta práctica promueve retomar viejos paradigmas de trabajo en donde el ingeniero/programador/investigador, estaba a cargo de todo el proceso de desarrollo: desde el diseño y la programación hasta la generación de pruebas, artefactos de software e instalación final. DevOps ha ganado mucho terreno gracias a la computación en la nube en donde en lugar de estar a cargo de recursos físicos, se está a cargo de recursos virtuales por lo que los roles del programador que solo programa, el ingeniero de pruebas que solo prueba software y finalmente el administrador de servidores que solamente se dedica a configurar hardware, se motivan a ser re-pensados. En ambientes como el de la computación en la nube, la infraestrutura está en el código lo que hace que al trabajar con este tipo de aplicaciones se tiene que estar consciente de muchas más cosas. De esta forma se puede decir que DevOps es la combinación entre desarrollo y operaciones. 

A pesar de ser un enfoque de desarrollo aún más reciente que las metodologías ágiles, DevOps ha sido influenciado por muchos de los principios de estas, principalmente del \emph{eXtreme Programming}. DevOps es visto por muchos como una extensión natural de las prácticas ágiles\cite{henrik-b} y actualmente se puede encontrar mucha literatura disponible sobre la aplicación de metodologías ágiles junto con DevOps.

Sin embargo la rápida penetración y auge de estas prácticas viene con un precio a pagar. En los primeros años de adopción de las metodologías ágiles era muy difícil encontrar personal con las habilidades necesarias, esto forzaba tanto a empleados como a empleadores a buscar formas alternas de formación. Por otro lado, los graduados de carreras de tecnologías de información del momento tampoco contaban con tales conocimiento debido a que los planes de estudio abarcaban poco o nada de estas prácticas. En los círculos académicos no había mucho interés en desarrollar estos temas debido a que se consideraba que desarrollo ágil no tenía bases teóricas sino que más bien ha sido desarrollada a partir de la práctica y la experiencia\cite{hazzan-dubinsky}. Esto ha venido cambiando de forma substancial y ahora temas sobre metodologías ágiles se han incorporado a los planes de estudio de carreras de tecnologías de la información de formas diversas. Aún se acusan muchos retos asociados a la enseñanza de estas prácticas, tal y como se indica en la Sección \ref{sec:retos}.

Con el paso de los años se ha constatado que estas prácticas han dejado de ser una moda o una tendencia del momento y que paralelo a fomentar mejoras en aspectos propios de ingeniería, también promueven el trabajo en equipo y colaborativo, habilidades de gestión, normas éticas y emprendedurismo \cite{hazzan-dubinsky, hickey-salas}. 

Para apoyar la mejora de habilidades técnicas y la inserción laboral, varios enfoques de enseñanza de ingeniería de software utilizando prácticas ágiles han sido reportados, \cite{ding-yousef-yue, steghoger-et-al, scharlau, schroeder-et-al, cubric, haaranen-lehtinen, kropp-meier-2}. Estas iniciativas persiguen exponer a los estudiantes a ambientes de trabajo similares a los que se encontraran en su práctica profesional y a la mejora de sus habilidades en desarrollo de software. Cobra aún mayor importancia la exposición de estos temas en la universidad si se toma en cuenta el gran auge de la computación en la nube. Baker \cite{advance-it} señala que \emph{``Los estudiantes de hoy en día puede ser que nunca se enfrenten a un servidor físico, un switch de red o un dispositivo de almacenamiento durante su carrera profesional futura pero aún así van a estar construyendo más infraestructura de IT que todos sus predecesores combinados. Se necesita enseñar a los estudiantes en cómo crear infraestructuas usando códido y cómo trabajar juntos de forma colaborativa para gestionar aplicaciones distribuidas complejas''}. 


En Costa Rica, la industria de desarrollo de software ha tenido un gran crecimiento en los últimos 20 años \cite{cenfotec-2, prosic}. A pesar del crecimiento, las organizaciones reportan deficiencias en la formación de la fuerza laboral y en los planes de estudio que, según señalan, no están actualizados con las necesidades del sector \cite{prosic}. Con respecto a la enseñanza de prácticas ágiles, las iniciativas en \cite{trejos-1, salazar, mora-et-al-1, mora-et-al-2} dan a conocer enfoques en cómo se están introduciendo a los estudiantes en estos temas. Se destacan los enfoques de enseñanza orientados a resolución de problemas y proyectos.

Esta revisión de lectura está organizada de la siguiente forma: la Sección \ref{sec:estrategia} presenta la estrategia de revisión de lectura utilizada. La Sección \ref{sec:desarrollo-agil} introduce a conceptos de desarrollo ágil de software así como a las prácticas de desarrollo de mayor aceptación. La Sección \ref{sec:ensenanza} recoge evidencia sobre la experiencia en la enseñanza de prácticas ágiles de desarrollo, la Sección \ref{sec:costa-rica} da a conocer esta misma experiencia para Costa Rica. La Sección \ref{sec:relevancia-agile} da conocer la relevancia de las prácticas ágiles en la industria. En la Sección \ref{sec:problema-objetivos} se definen el problema de investigación y los objetivos de la misma. El artículo concluye en la Sección \ref{sec:conclusion}.


\section{Estrategia} \label{sec:estrategia}
% Revisar la parte de cubrir el cuerpo de conocimiento relacionado con enseñanza de prácticas ágiles de desarrollo de software en las universidades
Esta sección presenta la estrategia emprendida para cubrir el cuerpo de conocimiento relacionado con enseñanza de prácticas ágiles de desarrollo de software en universidades. La estrategia general se basó en un proceso iterativo de identificación y lectura de artículos, luego identificar y leer artículos relevantes a partir de referencias y citas bibliográficas.

\subsection{Identificación de Preguntas de Investigación}
Las preguntas de investigación seleccionadas para conducir la revisión de la literatura fueron:
\begin{enumerate}
    \item ¿Cuáles son las prácticas de desarrollo ágil con mayor aceptación y qué relevancia tienen en la industria?
    \item ¿Se podría considerar a DevOps como una práctica de desarrollo ágil o representa más bien un enfoque diferente?
    \item ¿Qué información hay disponible acerca de de la enseñanza de prácticas ágiles de desarrollo de software carreras universitarias de TIC's en Costa Rica como en el extranjero? 
    \item ¿Cuáles son los beneficios y retos reportados en la enseñanza de estas prácticas de desarrollo ágil?
\end{enumerate}

\subsection{Estrategia de búsqueda} \label{sec:estrategia-busqueda}
Con el fin de identificar el primer conjunto de artículos relevantes, se hizo una revisión preliminar con Google Scholar\footnote{\url{http://scholar.google.com}} porque con este motor de búsqueda se puede abarcar una amplio número de artículos y actas académicas de diferentes fuentes. El criterio de búsqueda se basó en búsquedas de palabras derivadas del tema de investigación y las preguntas de investigación. Se incluyeron palabras como ``agile development'', ``agile engineering practices'', ``agile teaching'', ``devops education'', ``computer science education'' y ``software engineering education''. 

La búsqueda de la literatura se realizó en Setiembre del 2017 usando las siguientes bases de datos electrónicas:
\begin{itemize}
    \item ACM \emph{Digital Library} 
    \item IEEE \emph{Explore Digital Library}
    \item Safari Books Online\footnote{\url{https://www.safaribooksonline.com}}
    \item Google Scholar
\end{itemize}

\subsection{Criterio de selección de artículos}
Se aplicó el siguiente criterio de inclusión de artículos para esta revisión:
\begin{itemize}
    \item Estudios que den a conocer prácticas ágiles de desarrollo de software y 
    \item Estudios sobre la relación de prácticas ágiles de desarrollo de software y DevOps 
    \item Estudios que proporcionen algún tipo de solución, guía o marco de trabajo relacionado con la enseñana de prácticas ágiles de desarrollo de software en las universidades
    \item Estudios que reporten sobre el éxito, fracaso y retos de la experiencia de la enseñanza de prácticas ágiles de desarrollo de software
    \item Estudios que proporcionen evidencia sobre la enseñanza de prácticas ágiles de desarrollo de software y el impacto en la industria
\end{itemize}

El criterio de exclusión de artículos fue el siguiente:
\begin{itemize}
    \item Estudios relacionados en dar a conocer aspectos sobre procesos operativos y de negocios asociados con metodologías de desarrollo ágil.
    \item Estudios relacionados en la enseñanza de metodologías ágiles pero que no cubren o cubren muy poco aspectos sobre prácticas ágiles de desarrollo de software.
    \item Estudios relacionados con casos de estudio de la aplicación y/o enseñanza de prácticas ágiles de desarrollo de software fuera de la universidad.
    \item Artículos publicados hace más de 10 años atrás (rango aceptable 2007-2017).
\end{itemize}


\subsection{Resultado de la revisión} \label{sec:resultado-rev-lit}
Luego de obtener los artículos, literatura y recursos relevantes, se identificó que los mismos podían clasificarse en tres grupos: desarrollo ágil de software, enseñanza de prácticas ágiles de desarrollo de software en universidades extranjeras y enseñanza de prácticas ágiles de desarrollo de software en Costa Rica.


%¿Cuál es el nivel de penetración de prácticas ágiles de desarrollo de software en los planes de estudio de carreras en tecnología de la información (TIC's) en Costa Rica y qué factores propician o limitan esto?

\section{Desarrollo Ágil de Software} \label{sec:desarrollo-agil}
Las metodologías de desarrollo ágil de software emergen al final de de la década de los noventa. El término ``ágil'' se utiliza para agrupar una serie de métodos tales como \emph{Scrum}, \emph{eXtreme Programming}, Crystal, \emph{Feature Driven Development} (FDD), \emph{Dynamic Software Development Method} (DSDM) y \emph{Adaptative Software Development}\cite{rashina-et-al}. Las metodologías ágiles se caracterizan por promover el desarrollo interativo e incremental y la entrega frecuente de funcionalidad prioritaria para los clientes. Estas metodologías estan dirigidas a equipos pequeños y altamente colaborativos. \emph{Scrum} y XP son las metodologías que presentan mayores niveles de adopción\cite{version-one}: XP se centra en prácticas de desarrollo mientras que \emph{Scrum} cubre principalmente la administración del proyecto.

Las metodologías ágiles son adecuadas para proyectos con requerimientos altamente cambiantes, se motiva aceptar y responder ante los cambios. En 2001, los desarrolladores de varios de estas metodologías ágiles escribieron el Manifiesto Ágil \cite{agile-manifesto} (Apéndice \ref{apendice:a}). Los principios detrás del Manifiesto Ágil incluyen entrega rápida, frecuente consistente y continua de software funcional, respuesta a requerimientos cambiantes, comunicación efectiva y equipos motivados con capacidad de auto organizarse.

\subsection{Prácticas Ágiles de desarrollo de software} \label{sec:practicas-agiles-desarrollo}
De acuerdo con \cite{ford}, a pesar de la gran cantidad de literatura disponible sobre metodologías ágiles de software, mucha de la misma menciona poco acerca de las prácticas agiles de desarrollo de software. La mayoría se centran principalmente en gestión, procesos y estimación, y no tanto en la parte relacionada con la ingeniería de software.

En contraste con las metodologías, las prácticas ágiles están un nivel por debajo debido a estas ya que son una parte muy específica de una metodología, la cual aborda varios aspectos. Algunos ejemplos conocidos son programación en parejas, desarrollo orientado en pruebas y revisiones de código. A pesar de no haber una definición común en la literatura de práctica ágil \cite{diebold-dahlem}, se puede tomar XP en consideración para tener un mejor entendimiento por medio del estudio de la colección de prácticas de desarrollo que se promueven en esta metodología. A diferencia de otras metodologías como \emph{Scrum}, XP se dedica tanto a la gestión del proyecto como también en la forma en cómo los equipos construyen código \cite{shore-warden}.

Para efectos de esta revisión, se definen un conjunto de prácticas. Esto porque las metodologías ágiles llaman a sus prácticas de forma diferente. Para obtener un conjunto de prácticas de referencia, se tomó como punto de partida el reporte anual del estado de la metodologías ágiles de Version One \cite{version-one}, y el cual expone las prácticas ágiles en desarrollo de mayor aceptación. Las prácticas expuestas en este reporte se confrontaron con la literatura acerca de XP \cite{beck-andres, ford, shore-warden} con el fin de validar si se hace referencia de las mismas. El resultado es la siguiente lista de prácticas ágiles en desarrollo de software:
\begin{itemize}
    \item Pruebas
        \begin{itemize}
            \item Pruebas unitarias
            \item Desarrollo orientado a pruebas (TDD, por sus siglas en inglés)
            \item Desarrollo orientado al comportamiento (BDD, por sus siglas en inglés)
            \item Desarrollo orientado a pruebas de aceptación (ATDD, por sus siglas en inglés)
        \end{itemize}
        \item Integración continua
        \item Estándares de codificación
        \item Refactorización de código
        \item Puesta en producción continua
        \item Propiedad compartida del código
        \item Diseño emergente
\end{itemize}

En el apéndice \ref{apendice:b} se proveen definiciones de estas prácticas.

\subsection{DevOps}
DevOps es la práctica que combina desarrollo y operaciones y que desde el 2009 viene siendo adoptada por muchas organizaciones de la industria \cite{bang-et-al}. Se puede resumir como una práctica que anima a establecer un ambiente en donde la construcción, pruebas y la entrega de software puede realizarse de forma rápida, frecuente y más confiable \cite{henrik-b}. DevOps apareció como una respuesta directa a los retos de las plataformas de software a gran escala que se actualizan rápidamente, tal y como lo indica Anderson \cite{anderson}: \emph{``El movimiento DevOps emergió a partir de uno de los clásicos obstáculos en muchas organizaciones. Los desarrolladores construyen código y aplicaciones, y las envían al personal de operaciones solo para descubrir que el código y las aplicaciones no corren en producción. Es el clásico problema de ``se ejecutaba en mi máquina y funcionaba, ahora es problema del personal de operaciones'' ''}. 

\subsection{DevOps vs Metodologías Ágiles}
A pesar de no estar considerada como una metología de desarrollo ágil como tal, autores como Bæebark \cite{henrik-b} ven al DevOps como el siguiente paso natural del movimento de desarrollo ágil el cual hace énfasis en software funcional, colaboración, velocidad y respuesta al cambio. Sin embargo, DevOps tiene mayor énfasis en las operaciones e introduce elementos nuevos: el código nuevo, arreglo de errores e incrementos son desarrollados, probados, integrados y entregados a los usuarios finales en cuestión de horas, y el equipo de desarrollo es responsable total de los requerimientos, desarrollo, pruebas, entrega y monitoreo. En \cite{jabbari-et-al} se estudió la relación de las prácticas de DevOps con otras metodologías de desarrollo y se destaca con respeto a DevOps y a las metodologías ágiles que:
\begin{itemize}
    \item DevOps extiende las metodologías ágiles: los principios de DevOps puede proporcionar una extensión prágmatica a las prácticas ágiles. Se pueden lograr los objetivos de las metodologías ágiles al extender los principios de estas metodologías a través de una tubería de entrega de software.
    \item Las metodologías ágiles son facilitadores para DevOps: en \cite{jabbari-et-al} se menciona que las metodologías ágiles pueden ser consideradas como facilitadores para adoptar los principios detrás de DevOps.
    \item Las metodologías ágiles apoyan DevOps: al motivar la colaboración entre los miembros de los equipos, la automatización de la construcción, entrega y pruebas, el establecimiento de métricas y el intercambio de conocimientos y herramientas. 
\end{itemize}

De igual forma, para validar esta relación e interoperatibilidad entre ambos enfoques, la búsqueda de literatura realizada en \ref{sec:estrategia-busqueda} en Safari \emph{Books Online} retornó   329 títulos los cuales incluían libros y videos. Esto sugiere que hay mucha aceptación en el uso de ambos enfoques en conjunto.

En \cite{jabbari-et-al} también se expone que a pesar de las similitudes, DevOps no cumple con todos los principios propuestos en el Manifiesto Ágil \cite{agile-manifesto} y que DevOps elimina la brecha entre desarrolladores y personal de operaciones mientras que las metodologías ágiles están orientadas a alinear requirimientos de negocio con el desarrollo.

De acuerdo con \cite{jabbari-et-al}, las principales prácticas en desarrollo de software de DevOps son:
\begin{itemize}
    \item El código como infraestructura
    \item Administración de la configuración
    \item Integración continua
    \item Entrega continua
    \item Pruebas automatizadas
    \item Monitoreo de rendimiento
\end{itemize}
La guía en \cite{wiggins} es un ejemplo de la aplicación de las prácticas de DevOps identificadas en \cite{jabbari-et-al} adaptadas para desarrollo de aplicaciones de tipo software como servicio o SAAS\footnote{\emph{Software-as-a-service}} por sus siglas en inglés.

Al igual que en la sección \ref{sec:practicas-agiles-desarrollo}, en el apéndice \ref{apendice:b} se proveen definiciones de estas prácticas.

\section{Relevancia de Prácticas ágiles de desarrollo} \label{sec:relevancia-agile}
Los reportes consultados sobre el estado actual de metodologías ágiles\cite{version-one} y DevOps\cite{puppet-devops} reflejan una amplia aceptación de ambos enfoques en la industria de tecnología. Por ejemplo, en \cite{version-one} el 94\% de las organizaciones que tomaron parte del estudio respondieron afirmativamente a la pregunta de si practicaban enfoques ágiles y el 71\% de las mismas reportó que estan llevando a cabo iniciativas en DevOps. En \cite{puppet-devops} se señala que para lograr que las personas logren el máximo provecho de DevOps, las prácticas técnicas tales como automatización de pruebas, puesta en producción automática, integración continua y un manejo de versiones detallado son claves. Los que práctican DevOps tienen que estar concientes del uso de estas prácticas técnicas y no solamente de herramientas específicas. 

La encuesta realizada en \cite{kropp-meier-1} donde 160 compañías suizas y casi 200 profesionales en tecnología de información participaron, también muestra resultados claros. Las compañías y los profesionales que siguen metodologías ágiles están más satisfechos con los enfoques que estas promueven que con metodologías basadas en planeamiento. Otro resultado importante del estudio es que los principales objetivos de introducir desarrollo ágil son: la gestión de prioridades cambiantes y la mejora del proceso de desarrollo en general. La encuesta muestra que no hay tantos ingenieros con habilidades para desarrollo ágil y esto le sugiere a los autores que los estudiantes no están siendo educados con las habilidades necesarias. El 70\% de los participantes piensan que los bachilleres en computación tiene muy poco conocimiento de metodologías ágiles; la mayoría piensa lo mismo para los graduados de maestría. Cuando se consultó sobre si las metodologías ágiles deberían ser parte integral del plan de estudios de carreras de computación, la gran mayoría recomendó que el desarrollo de software ágil debería de ser parte integral de estos planes de estudio.

Un estudio similar se llevó a cabo en el estado de Minnesota, Estados Unidos, con el fin de evaluar una nuevas opciones en los planes de estudio de carreras de tecnología de información de las universidades de ese estado \cite{advance-it}. El estudio señala que las habilidades en desarrollo ágil son cada vez más importantes en las decisiones de contratación, el 62\% lo considera un factor relevante y el 41\% comenzaron a hacerlo en los últimos 3 años. El 59\% dice que su desarrollo y entrega de software están basadas en prácticas ágiles/DevOps/\emph{Cloud Computing}. Hay preocupación sobre la disposición de la mano de obra de Minnesota para promover una \emph{transformación digital} y de la capacidad del sistema educacional del estado en producir esta fuerza de trabajo: solamente el 32\% está de acuerdo que la fuerza de trabajo está bien preparada en términos de competencias y el 49\% no están de acuerdo en que el sistema educativo está produciendo una fuerza laboral digital suficientemente grande.

Los resultados de la revisión sistemática de la literatura realizada por Rademarcher y Walia \cite{radermacher-walia} para determinar qué áreas de los estudiantes graduados con más frecuencia no alcanzan las expectativas de la industria o la academia indican que los estudiantes recién graduados presentan carencias en áreas técnicas, personales y profesionales. Dentro de las carencias en las áreas técnicas, el estudio de \cite{radermacher-walia} señala las siguientes:
\begin{itemize}
    \item Carencias en herramientas de software: una categoría que reporta muchas deficiencias y que muestra varios ejemplos específicos. El ejemplo más comunmente reportado es la gestión de la administración de la configuración. Los estudios consultados en este artículo encontraron que los desarrolladores de software recién contratados carecían de experiencia en herramientas de gestión de la configuración. Así mismo se mencion que existe una brecha grande entre las habilidades de los profesionales de software con respecto a su educación.
    \item Pruebas: es una área que presenta deficiencias tanto en la industria como en la academia. Estudiantes avanzandos mostraron poca habilidad de uso de herramientas para cobertura de de pruebas y en escribir pruebas relevantes. Los desarrolladores de software recién contratados muestran carencias en la habilidad de llevar a cabo pruebas de código exhaustivas. También se reportaron problemas con respeto al \emph{debugging} de código.
\end{itemize}


\subsubsection{En Costa Rica}
El sector de tecnología de la información en Costa Rica ha experimentado el mismo crecimiento explosivo con respecto a la oferta y demanda de personal calificado que se ha llegado a reportar en otros países. De 1997 al 2000 las compañías de desarrollo de software y otros emprendimientos crecieron a una tasa del 40\% al 60\% en la planilla\cite{cenfotec-2}.

De acuerdo con \cite{prosic} con respecto a la fuerza laboral total del país, el empleo en el sector de tecnologías de la información representó en 2006 un $2,32\%$ y en 2007, $2,47\%$, mostrando un leve aumento de $0.15\%$ para el último año. El aumento de $4886$ trabajadores en el sector de tecnologías de la información representa un crecimiento del $10.7\%$ en relación con la fuerta laboral todal del sector. Por otro lado, el comportamiento del empleo en el área de producción de bienes y servicios de las tecnologías de la información, entre 2006 y 2007 respecto de la fuerza laboral, fue inverso: el porcentaje de trabajadores dedicados a la producción disminuyó un $19.8\%$, mientras que el porcentaje de empleados dedicados a servicios aumentó un $19.6\%$.

En \cite{murillo-trejos} se hizo un mapeo de las habilidades consideradas las más importantes a encontrar en profesionales en tecnologías de la información. Una de las motivaciones de este mapeo es dar a conocer las áreas en las que las empresas multinacionales están más interesadas y de esta forma motivar la creación de nuevos programas educativos para estimular esas habilidades. Las categorías que reportaron mayor interés en las multinacionales fueron las relacionadas con ``Desarrollo e Implementación'' y ``Entrega y Operación''. Esto refleja que las hay un interés marcado en contar con profesionales que puedan tener habilidades que cubran todas las necesidades que se presenten durante el ciclo de desarrollo de sistemas.


El estudio llevado a cabo por el Programa Sociendad de la Información y el Conocimiento \cite{prosic}(Prosic) de la Universidad de Costa Rica señala que hay una preocupación en el país por la escasez de mano de obra calificada para el sector de las tecnologías de la información, particularmente en la industria del software. Se añade que \emph{``en el país no se han tomado las medidas necesarias para que este sector se consolide dentro de una economía del conocimiento. Esta consolidación requiere, entre otras cosas, mayores inversiones en investigación y desarrollo y
adecuar los sistemas educativos existentes en el país para que preparen profesionales en adecuada cantidad, así como con buena calidad para satisfacer los requerimientos del mercado laboral.''}

Dentro del mismo estudio, se aplicó una encuesta a 160 organizaciones de tecnologías de la información y se identificó que el principal problema de para el desarrollo del sector es la poca disponibilidad de recurso humano calificado. En este apartado la palabra ``calificado'' es motivo de diferenciación porque aunque se cuenta con fuerza laboral, en muchos casos se ha logrado experimentar insuficiencia en la búsqueda de los conocimientos que se desea que posean. Se acusa de un rezago educativo y recomiendan:
\begin{itemize}
    \item Fortalecimiento de la educación técnica
    \item Ajuste de los programas: actualización de formación y curricula
    \item Actualización de docentes
    \item Diseño de carreras efectivamente articuladas
    \item Integración de oferta y demanda
    \item Integración entre empresas y sector educativo para la definición de perfiles de formación
\end{itemize}


\section{Enseñanza de Prácticas Ágiles de Desarrollo de Software} \label{sec:ensenanza}
De acuerdo con los resultados obtenidos a partir la búsqueda en la sección \ref{sec:resultado-rev-lit}, se evidencia que la enseñanza de prácticas ágiles de desarrollo es un tema activo de investigación y desarrollo en universidades extranjeras. En \cite{kropp-meier-1} se señala que a pesar que el desarrollo de software ágil ha existido por más de una década, incluso antes del Manifiesto Ágil, la enseñanza de desarrollo de software ágil sólo ha llamado la atención en conferencias educativas y de investigación en años recientes. Una razón para esto puede ser que el desarrollo ágil no tiene bases teóricas sino que más bien ha sido desarrollada a partir de la práctica. En \cite{hazzan-dubinsky} se discute las razones por las cuáles los programas de ingeniería de software deberían de enseñar desarrollo ágil de software. Enfatizan que los ingenieros de software no solamente necesitan ejercitar habilidades técnicas sino también sociales y éticas, que son la aspectos básicos del desarrollo ágil.


Aunque las formas y aplicación de la enseñanza de estos temas varían de institución en institución, en la literatura prevalece la opinión de que la mejor forma de aprender desarrollo ágil es mediante el enfoque de aprender-haciendo (\emph{learn by doing}). Se encontraron ejemplos en donde se introduce al aprendizaje en prácticas de desarrollo ágil por medio de la implementación de algún trabajo final de graduación\footnote{\emph{Capstone project}}\cite{ding-yousef-yue}, proyectos/casos de estudio que se desarrollan durante un semestre\cite{steghoger-et-al}, desarrollando juegos\cite{scharlau}, por medio de laboratorios de robótica \cite{schroeder-et-al}, Wikis \cite{cubric} o bien como un tema paralelo al principal de un curso. Por ejemplo en \cite{haaranen-lehtinen}, se introduce a los estudiantes al versionamiento de código como herramienta de apoyo al desarrollo de una aplicación Web. De acuerdo al estudio en \cite{kropp-meier-2} el dominio de habilidades técnicas y prácticas de ingeniería representan la base para el entendimiento de la aplicación de otros temas sobre metodologías ágiles y para ir desarrollando una mentalidad dirigida al desarrollo de software de calidad.

Con respecto a la enseñanza de DevOps en universidades, se siguen un enfoques similares a los anteriormente mencionados para las prácticas de desarrollo ágil pero con un mayor énfasis en la introducción en temas de automatización \cite{henrik-b, bang-et-al, betz-et-al} de procesos de construcción, pruebas, entrega y computación en la nube. Un caso interesante es el que se expone en \cite{hickey-salas}, ``un campo de entrenamiento para emprendedores'', el cual es básicamente un curso de verano dedicado a ejercitar al máximo las habilidades de programación de los estudiantes y por medio de esto ejercitar también prácticas ágiles de desarrollo y de DevOps.





\subsection{Prácticas Ágiles de Desarrollo en Planes de Estudio de Referencia}
Desde la década de los sesenta, la \emph{American Computing Machinery} (ACM) junto con sociedades profesionales y científicas en computación se han dado a la tarea de propoponer y adaptar  recomendaciones en los planes de estudio debido al panorama cambiante de la tecnología\cite{acm-curriculum}.

En la guía curricular recomendada para los programas de bachillerato para la carrera de ciencias de la computación en \cite{acm-cs-curriculum} se incluyen temas tales como integración contínua, automatización, control de versiones, estándares de codificación, desarrollo orientado a pruebas, pruebas unitarias y pruebas de integración como parte del cuerpo de conocimiento de ingeniería de software sugerido para esta carrera. Para la guía curricular recomendada para los programas de ingeniería de software \cite{acm-se-curriculum} se incluyen temas de prácticas ágiles de desarrollo de software en las secciones de Validación y Verificacion, y en la sección de Procesos asociados al desarrollo de software. También se pudo constatar la prescencia de prácticas ágiles de desarrollo en cursos de referencia sobre  programación, ingeniería de requerimientos, proyecto finales, entre otros, que han desarrollado universidades en el extranjero. En \cite{acm-msis-curriculum} las prácticas ágiles de desarrollo aparecen como parte de las competencias del área de desarrollo y entrega de sistemas. 

Los autores de \cite{betz-et-al} junto con un grupo de trabajo formado por investigadores de universidades del estado de Minnesota en Estados Unidos, proponen un plan de estudios para carreras de sistemas de información y de tecnologías de la información que responde a las tendencias de desarrollo ágil y DevOps \cite{advance-it} de la actualidad. La propuesta que presentan fue la que se pudo identificar como la más radical hacia en la enseñanza y aprendizaje de prácticas ágiles de desarrollo.

\section{Enseñanza de Prácticas Ágiles en Costa Rica} \label{sec:costa-rica}
La evidencia recolectada apunta a que los primeras iniciativas dirigidas a la introducción y enseñanza de prácticas ágiles en las universidades de Costa Rica se dio a inicios del 2000. El siguiente fragmento tomado de \cite{cenfotec-2} reseña muy bien la situación de ese momento y da a conocer el porqué las universidades en Costa Rica empezaron a revisar sus planes de estudio para que fueran más acordes con la tendencias de la industria: \emph{``A fines de la década de 1990, la demanda de profesionales y técnicos en Informática tenía un crecimiento cercano al 90\% interanual Norteamérica, Europa Occidental y Japón. En Costa Rica, la industria costarricense de software crecía, en personal, entre un 40\% y 60\% anualmente, gracias a su éxito en mercados internacionales. La sostenibilidad de esa industria radica no en su costo, sino en la calidad de sus productos, la cual depende fundamentalmente del talento humano y de los sistemas de trabajo (procesos) aplicados. El cambio en la tecnología de software y el aumento en la demanda de aplicaciones informáticas – particularmente para la Web y los dispositivos móviles – han exigido innovaciones en la forma de educar profesionales en desarrollo de software preparados para producir aplicaciones y componentes de software de calidad mundial. Además, desde el año 2003, Costa Rica participa como proveedor de servicios tecnológicos y empresariales habilitados por tecnologías digitales. Esto ha planteado la necesidad de preparar profesionales con perfiles distintos de los que se graduaban de las universidades costarricenses.''}

En sus inicios, la hoy Universidad Cenfotec \cite{cenfotec-1} presentó un plan de estudios para el programa de Especialista en Tecnología de Software, el cual se diferenciaba de los carreras universitarias convenciales al sacrificar la generalidad en aras de la especialidad y permitir al graduado incorporarse al mercado laboral para realizar desarrollo de software de alta calidad. La intención detrás de este programa era ejercitar de forma intensiva planificación, diseño y programación de software a partir de la ejecución de proyectos y del enfoque de aprender-haciendo.

La Universidad de Costa Rica ha reportado la introducción de \emph{Scrum} y programación extrema como parte de sus cursos en ingeniería de software. En \cite{salazar} se reporta el uso de estas prácticas junto con RUP. El profesor a cargo establece un proyecto que se va desarrollando durante todo el semestre. La Universidad Nacional reporta la ejecución de iniciativas para impartir cursos de programación bajo modelos pedagógicos orientados a proyectos y a resolución de problemas \cite{mora-et-al-1, mora-et-al-2}, sin embargo en este caso, la experiencia reportada no incluye prácticas ágiles de desarrollo sino más bien lo que se intenta es propiciar el aprendizaje colaborativo, la participación y la autonomía. Cenfotec también promueve este enfoque de aprendizaje colaborativo y orientado a resolución de problemas\cite{trejos-1, cenfotec-2} por medio proyectos integradores los cuales aparte de formar habilidates técnicas o fuertes, también pretenden formar en habilidades suaves como comunicación, creatividad, valores éticos, agilidad, educación continua, entre otros.

\subsection{Sobre la experiencia en la enseñanza de prácticas ágiles de desarrollo}

\subsection{Beneficios}
\begin{itemize}
    \item Uso de herramientas de software de actualidad y un aprendizaje más alineado con las tendencias de la industria\cite{cenfotec-2}.
    \item Representa para muchos estudiantes la primera experiencia en donde se codifica, prueba, integra y se pone en producción un sistema, recibiendo en cada paso retroalimentación inmediata\cite{kropp-meier-2}.
    \item En \cite{kropp-meier-1} se reporta una mejora de las habilidades de programación y calidad del código.
    \item De acuerdo a \cite{scharlau} los estudiantes sentían que se rompía la rutina y que se daba mayor paso a la espontaneidad y la innovación. Esto también propiciaba que los estudiantes tuvieran mayor compromiso en los proyectos en los que trabajaron.
    \item La actividad en cuanto a los emprendimientos llevados a cabo por los estudiantes aumenta considerablemente luego de ser introducidos a estos temas\cite{hickey-salas}.
    \item Repetir la misma metodología y herramientas en cada \emph{sprint} mejoró la productividad de desarrollo y esto revela una mejoría de la curva de aprendizaje. Conforme fueron avanzando el desarrollo de los \emph{sprints}, las horas requeridas de esfuerzo para desarrollar una funcionalidad fue disminuyendo, por lo tanto la productividad mejoró\cite{salazar}.
    \item La introducción de prácticas ágiles de desarrollo fortaleció el proceso de pruebas, asegurando mayor calidad y evitando el re-trabajo\cite{salazar}
    \item Las prácticas ágiles de desarrollo de software, al ser el aspecto más técnico en las metodologías ágiles resulta ser también el más atractivo para los estudiantes y en donde concentran más sus esfuerzos de aprendizaje\cite{steghoger-et-al}
    \item La mayoría de la evidencia señala que la aplicación de la enseñanza de prácticas ágiles se hace a través de enfoques orientados a resolución de problemas y proyectos. Los resultados obtenidos en \cite{mora-et-al-1} resaltan la importancia de estos enfoques porque ofrecen a los estudiantes mayores posibilidades de tomar decisiones cuando se trabaja con la solución de problemas, y en la forma de abordarlo y, como consecuencia, aumentar su motivación en el proceso de aprendizaje y de asumir mayor responsabilidad.
    \item Hazzan y Dubinsky \cite{hazzan-dubinsky} señalan otros beneficios asociados a la enseñanza de prácticas ágiles de desarrollo tales como:
    \begin{itemize}
        \item Su aplicabilidad en la industria
        \item Trabajo en equipo
        \item Apoya otros procesos de aprendizaje
        \item Lidia con aspectos humanos, promueve la diversidad, habilidades de gestión y normas éticas
        \item Promueve hábitos mentales
        \item Brinda una imagen detallada de ingeniería de software
        \item Las metodologías ágiles proporcionan un marco de trabajo que se puede enseñar
    \end{itemize}
\end{itemize}

\subsection{Retos} \label{sec:retos}
\begin{itemize}

    \item Las instituciones educativas son lentas en crear e innovar en los planes de estudio para preparar a la futura fuerza laborar y desarrollar programas para re-entrenar aquellos que se ya se encuentran trabajando \cite{cenfotec-2}.
    \item Existen muy pocos artículos publicados acerca de la enseñanza de DevOps\cite{henrik-b}.
    \item En \cite{steghoger-et-al} se señala que:
        \begin{itemize}
            \item Cuando se enseñan procesos de desarrollo ágil debe de haber retroalimentación rápida y directa por parte de los profesores. La retroalimentación permite la corrección temprana a prácticas que se aplican mal.
            \item La planificación de cursos en estos temas requiere de un gran esfuerzo de adquisición de conocimiento previo, maxime si los potenciales instructores han tenido poca exposición a los mismos.
            \item La enseñanza de metodologías ágiles necesita tener un elemento práctico, se necesita aplicar para que se pueda saber. Cursos en donde se exponen solamente aspectos teóricos resultan menos atractivos para los estudiantes. 
        \end{itemize} 
        
        \item Si la formación en tecnologías ágiles y DevOps están por fuera de los planes regulares y se brinda en la modalidad de curso de verano o electivo, esto puede llegar a necesitar de mucho esfuerzo por parte de unidad académica a cargo y reportar pocas ganancias\cite{hickey-salas}.
        \item No es conveniente que se aprendan y apliquen metodologías ágiles como único método de enseñanza en cursos de ingeniería de software, porque generalmente estos modelos de desarrollo promueven un proceso informal, con escasa documentación y con un mínimo de productos de trabajo de ingeniería de software que atentan con la calidad del software y otras prácticas recomendadas\cite{salazar}
        \item En \cite{mora-et-al-1} se recalca la importancia de contar con personal motivado y comprometido en adoptar nuevos cambios en los cursos de la carrera: \emph{``Además, aunque la mayoría de los docentes manifiestan estar abiertos a hacer cambios que mejoren el aprendizaje y motiven más a los estudiantes, es evidente que estos procesos implican más trabajo y un cambio de cultura de los actores involucrados. Además es claro que se requiere de una línea de gestión clara y de una estructura de gestión curricular flexible que permita modificar las estructuras de evaluación tradicionales''}.
        \item Kropp y Meier \cite{kropp-meier-1} opinan que los cursos sobre desarrollo ágil de software no pueden ser enseñados en cursos aislados de ingeniería de software. Un reto es la integración de prácticas de desarrollo ágil en otros cursos como programación o análisis y diseño orietado a objetos, algoritmos y estruturas de datos.
\end{itemize}


\section{Definición del Problema y Objetivos} \label{sec:problema-objetivos}

\subsection{Problema}
Las prácticas ágiles de desarrollo de software están cambiando las formas tradicionales de hacer y entregar software. Estas prácticas gozan de una gran aceptación y auge dentro la industria del software desde hace más de una década atrás. En Costa Rica, a pesar de contar con una creciente industria de software, encuestas y estudios revelan que los empleadores señalan deficiencias en las habilidades de desarrollo ágil como programación, pruebas, integración y entrega. Aunado a lo anterior se acusa un rezago en los planes de estudio de las carreras de tecnología de la información con respecto a las necesidades de la industria.

\subsection{Objetivo General}
Explorar la penetración de prácticas ágiles de desarrollo de software en los planes de estudio de las carreras en tecnología de la información en Costa Rica.

\paragraph{Pregunta de Investigación asociada} ¿Cuál es el nivel de penetración de prácticas ágiles de desarrollo de software en los planes de estudio de carreras de tecnología de la información en Costa Rica y qué factores propician o limitan esto?


\subsection{Objetivos Específicos}

\subsubsection{Dar a conocer las prácticas ágiles de desarrollo de software de mayor adopción en la industria.} \label{sec:objetivo-especifico-1}

\subsubsection{Identificar la aplicación de prácticas ágiles de desarrollo de software como parte de la enseñanza de carreras de tecnología de información en Costa Rica.} \label{sec:objetivo-especifico-2}

\subsubsection{Exponer causas por medio de las cuales la enseñanza de prácticas ágiles de desarrollo de software se ve impactada tanto de forma positiva como negativa en los centros de estudio en donde se imparte carreras de tecnología de la información en Costa Rica.} \label{sec:objetivo-especifico-3}

\subsection{Preguntas de investigación asociadas a los objetivos específicos}
\begin{itemize}
    \item ¿Qué son prácticas ágiles de desarrollo de software? (Objetivo específico \ref{sec:objetivo-especifico-1})
    \item ¿Cuáles son las prácticas ágiles de desarrollo de software que gozan de mayor aceptación en la industria? (Objetivo específico \ref{sec:objetivo-especifico-1})
    \item ¿Cuáles prácticas ágiles de desarrollo de software se enseñan durante la carrera? (Objetivo específico \ref{sec:objetivo-especifico-2})
    \item ¿Cómo se implementan estas prácticas de desarrollo ágil en los planes de estudio? (Objetivo específico \ref{sec:objetivo-especifico-2})
    \item ¿Qué causas benefician o perjudican la enseñanza de estas prácticas en las universidades en donde se imparte? (Objetivo específico \ref{sec:objetivo-especifico-3})
\end{itemize}


\section{Conclusión} \label{sec:conclusion}
Las prácticas ágiles de desarrollo de software se han logrado posicionar como una guía mediante la cual se puede obtener mejores resultados durante la construcción del software. Como se señala en la literatura consultada, han dejado de dejado de ser una mera tendencia y han abierto el camino hacia nuevas enfoques como lo es DevOps. Durante los primeros años de su aparición, la hubo escasa exposición a introducción de las prácticas así como de metodologías ágiles en general, el personal académico le tomó poco interés en parte motivado por el hecho de saber si esto estos enfoques de desarrollo iban a ser una moda pasajera o bien, si su implementación reportaba beneficios a tomar en cuenta para su adopción.

Hoy en día como se indica en las encuestas y reportes consultados, las prácticas ágiles de desarrollo siguen ostentando de gran popularidad y aceptación dentro de la industria del software. Se continua reportando un aumento de su uso y se considera que al poner en ejercicio estas prácticas se promueve a las empresas y proyectos de software dar el salto a esquemas de trabajo que mejoran y hacen más rápido cubrir iteraciones en el ciclo de desarrollo de sistemas como lo es DevOps que es una combinación entre desarrollo y operaciones, el cual está altamente influenciado por principos de desarrollo ágil, particularmente por \emph{eXtreme Programming}.

Universidades reportan la aplicación de la enseñanza de las prácticas ágiles de desarrollo dentro de los planes de estudio de carreras de tecnología de la información de distintas maneras pero se nota que a pesar de sus diferencias, todas estas formas son variantes del enfoques de aprendizaje basado en resolución de problemas y proyectos, enfoques de aprender-haciendo. No obstante, la enseñanza y aprendizaje de estos temas sigue presentando una serie de retos que van desde la falta de literatura disponible hasta cambios en la cultura del profesorado.

En Costa Rica, también se ha reportado la enseñanza de prácticas ágiles de desarrollo en las universidades. Se destacan los esfuerzos de la Universidad Cenfotec que desde inicios del 2000 incluyó principios ágiles como parte de los proyectos de programación que llevaban a cabo los estudiantes. Los enfoques de aprender-haciendo también son los que se utilizan a la hora de introducir al estudiante en estos temas. 

A pesar de la relevancia y adopción de las metodologías ágiles en la industria de software en Costa Rica, los empleadores mantienen opiniones no tan favorables como se podría esperar en cuanto a las habilidades como programación, pruebas, integración y entrega de los empleados y se aduce que una razón de esto es que los planes de estudio de las universidades muestran un rezago en con respecto a las necesidades de la industria. Lo anterior sugiere que una revisión en la adopción de prácticas ágiles de desarrollo de software en los planes de estudio de las carreras de tecnología de la información de Costa Rica con el fin de explorar cuáles prácticas se están enseñando y cómo se están aplicando las mismas. Esto podría identificar oportunidades de mejora en las mismas con el fin de pulir las habilidades en programación de los estudiantes y hacer que el conocimiento adquirido pueda estar mejor alineado con las tendencias de la tecnología y la industria.



% An example of a floating figure using the graphicx package.
% Note that \label must occur AFTER (or within) \caption.
% For figures, \caption should occur after the \includegraphics.
% Note that IEEEtran v1.7 and later has special internal code that
% is designed to preserve the operation of \label within \caption
% even when the captionsoff option is in effect. However, because
% of issues like this, it may be the safest practice to put all your
% \label just after \caption rather than within \caption{}.
%
% Reminder: the "draftcls" or "draftclsnofoot", not "draft", class
% option should be used if it is desired that the figures are to be
% displayed while in draft mode.
%
%\begin{figure}[!t]
%\centering
%\includegraphics[width=2.5in]{myfigure}
% where an .eps filename suffix will be assumed under latex, 
% and a .pdf suffix will be assumed for pdflatex; or what has been declared
% via \DeclareGraphicsExtensions.
%\caption{Simulation results for the network.}
%\label{fig_sim}
%\end{figure}

% Note that IEEE typically puts floats only at the top, even when this
% results in a large percentage of a column being occupied by floats.


% An example of a double column floating figure using two subfigures.
% (The subfig.sty package must be loaded for this to work.)
% The subfigure \label commands are set within each subfloat command,
% and the \label for the overall figure must come after \caption.
% \hfil is used as a separator to get equal spacing.
% Watch out that the combined width of all the subfigures on a 
% line do not exceed the text width or a line break will occur.
%
%\begin{figure*}[!t]
%\centering
%\subfloat[Case I]{\includegraphics[width=2.5in]{box}%
%\label{fig_first_case}}
%\hfil
%\subfloat[Case II]{\includegraphics[width=2.5in]{box}%
%\label{fig_second_case}}
%\caption{Simulation results for the network.}
%\label{fig_sim}
%\end{figure*}
%
% Note that often IEEE papers with subfigures do not employ subfigure
% captions (using the optional argument to \subfloat[]), but instead will
% reference/describe all of them (a), (b), etc., within the main caption.
% Be aware that for subfig.sty to generate the (a), (b), etc., subfigure
% labels, the optional argument to \subfloat must be present. If a
% subcaption is not desired, just leave its contents blank,
% e.g., \subfloat[].


% An example of a floating table. Note that, for IEEE style tables, the
% \caption command should come BEFORE the table and, given that table
% captions serve much like titles, are usually capitalized except for words
% such as a, an, and, as, at, but, by, for, in, nor, of, on, or, the, to
% and up, which are usually not capitalized unless they are the first or
% last word of the caption. Table text will default to \footnotesize as
% IEEE normally uses this smaller font for tables.
% The \label must come after \caption as always.
%
%\begin{table}[!t]
%% increase table row spacing, adjust to taste
%\renewcommand{\arraystretch}{1.3}
% if using array.sty, it might be a good idea to tweak the value of
% \extrarowheight as needed to properly center the text within the cells
%\caption{An Example of a Table}
%\label{table_example}
%\centering
%% Some packages, such as MDW tools, offer better commands for making tables
%% than the plain LaTeX2e tabular which is used here.
%\begin{tabular}{|c||c|}
%\hline
%One & Two\\
%\hline
%Three & Four\\
%\hline
%\end{tabular}
%\end{table}


% Note that the IEEE does not put floats in the very first column
% - or typically anywhere on the first page for that matter. Also,
% in-text middle ("here") positioning is typically not used, but it
% is allowed and encouraged for Computer Society conferences (but
% not Computer Society journals). Most IEEE journals/conferences use
% top floats exclusively. 
% Note that, LaTeX2e, unlike IEEE journals/conferences, places
% footnotes above bottom floats. This can be corrected via the
% \fnbelowfloat command of the stfloats package.






% if have a single appendix:
%\appendix[Proof of the Zonklar Equations]
% or
%\appendix  % for no appendix heading
% do not use \section anymore after \appendix, only \section*
% is possibly needed

% use appendices with more than one appendix
% then use \section to start each appendix
% you must declare a \section before using any
% \subsection or using \label (\appendices by itself
% starts a section numbered zero.)
%


\appendices
\section{El Manifiesto Ágil}\label{apendice:a}

\begin{center}
\emph{Individuos e interacciones sobre procesos y herramientas
Software funcionando sobre documentación extensiva
Colaboración con el cliente sobre negociación contractual
Respuesta ante el cambio sobre seguir un plan}
\end{center}

\emph{Esto es, aunque valoramos los elementos de la derecha,
valoramos más los de la izquierda.}

% you can choose not to have a title for an appendix
% if you want by leaving the argument blank
\section{Definiciones sobre Prácticas Ágiles}\label{apendice:b}

\paragraph{Pruebas unitarias} las pruebas unitarias se concentran en una clase o método. Se ejecutan enteramente en memoria lo que las hace muy rápidas\cite{shore-warden}.

\paragraph{Desarrollo orientado a pruebas (TDD)} es un ciclo rápido de prueba, codificación y refactorización\cite{shore-warden}. La idea es qeu cuando se desarrolla una nueva pieza de funcionalidad o se arregla un error, los desarrolladores primero crean una prueba que actúa como una especificación ejecutable del comportamiento esperado del código que se va a escribir. Estas pruebas no solamente orientan el diseño de la aplicación sino que también sirven tanto como prueba de regresión y como documentación del código y del comportamiento esperado de la aplicación.\cite{humble}

\paragraph{Desarrollo orientado al comportamiento (BDD)} toma la posición de que puede convertir una idea de un requisito en código implementado, probado y listo para la producción de manera simple y eficaz, siempre y cuando el requisito sea lo suficientemente específico para que todos los involucrados sepan lo que está pasando. BDD utiliza una historia como la unidad básica de funcionalidad, y por lo tanto de entrega. Los criterios de aceptación son una parte intrínseca de la historia.\cite{michael-b}. 

\paragraph{Desarrollo orientado a pruebas de aceptación (ATDD)} es una forma avanzada de TDD en el cual los criterios de aceptación automatizados, definidos en colaboración con los usuarios, orientan y enfocan el proceso de desarrollo. Esto ayuda a asegurar que todo el mundo entienda qué características están en desarrollo.


\paragraph{Integración continua}  la idea que hay detrás es que si la integración regular del códiga es algo bueno integrar, ¿por qué no hacerlo todo el tiempo? En el contexto de la integración ``todo el tiempo'' significa cada vez que alguien promueva cualquier cambio al sistema de control de versiones\cite{humble}. 

\paragraph{Estándares de codificación} \emph{eXtreme Programming} recomienda la creacio4n de estándares y guías de codificación con las que los desarrolladores se puedan poner de acuerdo en adherise cuando programan\cite{shore-warden}. Esto incluye: hábitos de programación, ambientes de desarrollo integrados, herramientos, configuración de archivos, convenciones de construcción y diseño y manejo de errores.

\paragraph{Puesta en producción continua} técnica que alienta a poner en producción cada cambio que haya pasado a través proceso de pruebas automatizadas. Aunque también se puede hacer entregar de forma continua software funcional en ambientes de pruebas o pre-producción, el punto crucial con esta técnica que entregar software funcional directamente en ambientes de producción\cite{humble}.

\paragraph{Propiedad compartida del código} extiende la responsabilidad de mantenimiento del código a todos los programadores. Todos comparten la responsabilidad de la calidad del código. Ninguna persona reclama la propiedad de alguna parte del sistema y cualquiera puede hacer un cambio en cualquier lugar del código\cite{shore-warden}.

\paragraph{Diseño emergente} \emph{eXtreme programming} hace demandas desafiantes a los programadores: cada semana deben finalizar de 4 a 10 historias de valor para los clientes. Cada semana, los clientes pueden revisar el plan actual e introducir nuevas historias. En otras palabras, como programador se tiene que estar en la capacidad de producir valor al cliente de forma constante. Afortunadamente, \emph{eXtreme programming} provee una solución a esto: diseño emergente e incremental, que permite a construir nueva infraestructura técnica incrementalmente, en pequeñas piezas, de acuerdo a como se entregan las historias\cite{shore-warden}.

\paragraph{El código como infraestructura} es la práctica en donde técnicas, procesos y herramientas utilizadas en el desarrollo de software son utilizadas para administrar la entrega y configuración de aplicaciones. Una organización que carece de código como infraestructura puede sufrir de retrasos en la entrega, por lo general debido a que los cambios en la infraestructura crean diferencias entre los ambientes de desarrollo y lo que administran el personal de operaciones\cite{humble}.

\paragraph{Administración de la configuración} se refiere al proceso mediante el cual todos los artefactos relevantes a un projecto y las relaciones entre ellos son almacenados, obtenidos, identificados de forma única y modificados. La estrategia de administracio4n de la configuración determinará cómo se gestionan todos los cambios que pasan dentro de un proyecto. Por lo tanto, se registra la evolución de los sistemas y aplicaciones\cite{humble}.

\paragraph{Pruebas automatizadas} El reto detrás de las pruebas de software es que una tarea que es sumamente importane pero que a la vez puede llegar a consumir mucho tiempo. La respuesta a esto es aplicar un conjunto de pruebas automatizadas tales como pruebas unitarias, de integración, de carga y de aceptación las cuales se ejecutan automáticamente luego de que se efectúa algún cambio en la aplicación. Una prueba automatizada añade valor al encontrar defectos antes que el código sea utilizado por los usuarios finales\cite{humble}.

\paragraph{Monitoreo del rendimiento} lo que se pretende es brindar visibilidad del estado actual de la aplicación. Esto se puede poner a disposición por medio de interfaces gráficas las cuales muestran métricas, alertas y bitácoras. Al analizar estos resultados se puede ir hasta una línea de código y determinar si existe un problema en la misma o en la infraestructura\cite{humble}.



% use section* for acknowledgment
%\section*{Acknowledgment}
%The authors would like to thank...


% Can use something like this to put references on a page
% by themselves when using endfloat and the captionsoff option.
\ifCLASSOPTIONcaptionsoff
  \newpage
\fi



% trigger a \newpage just before the given reference
% number - used to balance the columns on the last page
% adjust value as needed - may need to be readjusted if
% the document is modified later
%\IEEEtriggeratref{8}
% The "triggered" command can be changed if desired:
%\IEEEtriggercmd{\enlargethispage{-5in}}

% references section

% can use a bibliography generated by BibTeX as a .bbl file
% BibTeX documentation can be easily obtained at:
% http://www.ctan.org/tex-archive/biblio/bibtex/contrib/doc/
% The IEEEtran BibTeX style support page is at:
% http://www.michaelshell.org/tex/ieeetran/bibtex/
%\bibliographystyle{IEEEtran}
% argument is your BibTeX string definitions and bibliography database(s)
%\bibliography{IEEEabrv,../bib/paper}
%
% <OR> manually copy in the resultant .bbl file
% set second argument of \begin to the number of references
% (used to reserve space for the reference number labels box)


\begin{thebibliography}{1}


%\bibitem{IEEEhowto:kopka}
%H.~Kopka and P.~W. Daly, \emph{A Guide to \LaTeX}, 3rd~ed.\hskip 1em plus
%  0.5em minus 0.4em\relax Harlow, England: Addison-Wesley, 1999.

\bibitem{li-jian-armin-eberlein}
Li~Jian y Armin~Eberlein. 2009. \emph{An analysis of the history of classical software development and agile development}. Proceedings of the 2009 International Conferencen on Systems, Man, and Cybernetics. San Antonio, TX, USA. Octubre 2009. DOI: \url{https://ezproxy.itcr.ac.cr:2878/10.1109/ICSMC.2009.5346888}

\bibitem{agile-manifesto}
Kent~Beck, Mike~Beelde, Arie~van Bennekum, Allistair Cockburn, et al. 2001. \emph{Manifesto for Agile Software Development}. Agile Alliance.

% 3
\bibitem{version-one}
Version One. 2017. \emph{11th Annual State of Agile Report}. Obtenido de \url{http://stateofagile.versionone.com/}

% 4
\bibitem{henrik-b}
Henrik Bærbak Christensen. 2016. \emph{Teaching DevOps and Cloud Computing using a Cognitive Apprenticeship and Story-Telling Approach}. In Proceedings of the 2016 ACM Conference on Innovation and Technology in Computer Science Education (ITiCSE '16). ACM, New York, NY, USA, 174-179. DOI: \url{https://ezproxy.itcr.ac.cr:2878/10.1145/2899415.2899426}

% 5
\bibitem{hazzan-dubinsky}
Orit Hazzan and Yael Dubinsky. 2007. \emph{Why software engineering programs should teach agile software development}. SIGSOFT Softw. Eng. Notes 32, 2 (March 2007), 1-3. DOI: \url{http://ezproxy.itcr.ac.cr:2075/10.1145/1234741.1234758} 

% 6
\bibitem{hickey-salas}
Timothy J. Hickey and Pito Salas. 2013. \emph{The entrepreneur`s bootcamp: a new model for teaching web/mobile development and software entrepreneurship}. In Proceeding of the 44th ACM technical symposium on Computer science education (SIGCSE '13). ACM, New York, NY, USA, 549-554. DOI:\url{http://ezproxy.itcr.ac.cr:2075/10.1145/2445196.2445361}

% 7
\bibitem{cubric}
Marija Cubric. 2008. \emph{Agile learning \& teaching with wikis: building a pattern}. In Proceedings of the 4th International Symposium on Wikis (WikiSym '08). ACM, New York, NY, USA, , Article 28 , 2 pages. DOI: \url{http://ezproxy.itcr.ac.cr:2075/10.1145/1822258.1822296}

% 8
\bibitem{ding-yousef-yue}
Dabin Ding, Mahmoud Yousef, and Xiaodong Yue. 2017. \emph{A case study for teaching students agile and scrum in Capstone course}. J. Comput. Sci. Coll. 32, 5 (May 2017), 95-101.

% 9
\bibitem{haaranen-lehtinen}
Lassi Haaranen and Teemu Lehtinen. 2015. \emph{Teaching Git on the Side: Version Control System as a Course Platform}. In Proceedings of the 2015 ACM Conference on Innovation and Technology in Computer Science Education (ITiCSE '15). ACM, New York, NY, USA, 87-92. DOI: \url{http://ezproxy.itcr.ac.cr:2075/10.1145/2729094.2742608}

% 10
\bibitem{kropp-meier-2}
Martin~Kropp, Andreas Meier. 2014. \emph{New sustainable teaching approaches in software engineering education}. IEEE Global Engineering Education Conference (EDUCON). IEEE, 2014, pp. 1019–1022. DOI: \url{https://ezproxy.itcr.ac.cr:2878/10.1109/EDUCON.2014.6826229}

% 11
\bibitem{scharlau}
Bruce A. Scharlau. 2013. \emph{Games for teaching software development}. In Proceedings of the 18th ACM conference on Innovation and technology in computer science education (ITiCSE '13). ACM, New York, NY, USA, 303-308. DOI: \url{http://ezproxy.itcr.ac.cr:2075/10.1145/2462476.2462494}

% 12
\bibitem{schroeder-et-al}
Andreas Schroeder, Annabelle Klarl, Philip Mayer, Christian Kroiß. 2012. \emph{Teaching agile software development through lab courses}. Global Engineering Education Conference (EDUCON), IEEE. DOI: \url{https://ezproxy.itcr.ac.cr:2878/10.1109/EDUCON.2012.6201194}

% 13
\bibitem{steghoger-et-al}
Jan-Philipp Steghöfer, Eric Knauss, Emil Alégroth, Imed Hammouda, Håkan Burden, and Morgan Ericsson. 2016. \emph{Teaching Agile: addressing the conflict between project delivery and application of Agile methods}. In Proceedings of the 38th International Conference on Software Engineering Companion (ICSE '16). ACM, New York, NY, USA, 303-312. DOI: \url{https://ezproxy.itcr.ac.cr:2878/10.1145/2889160.2889181}

% 14
\bibitem{advance-it}
Advance IT Minnesota.  2016.\emph{Renewing the IT Curriculum: Responding to Agile, DevOps, and Digital Transformation}. Rep. (November 1, 2016). St. Paul, MN. Obtenido de \url{www.DynamicIT.education}

% 15
\bibitem{cenfotec-2}
Cenfotec. 2016. \emph{``Currículos basados en proyectos integradores.pdf'' y póster ``Proyectos integradores en U Cenfotec.pdf''}. SIIC Simposio Internacional sobre Innovaciones Curriculares, UCR, 2016.

% 16
\bibitem{prosic}
Claudio Pinto, Rafael Herrera, Francisco Mata, Rosaura Matarrita y otros. 2009. \emph{Formación de capital humano en el sector de TIC en Costa Rica}. \hskip 1em plus 0.5em minus 0.4em\relax Facultad Latinoamericana de Ciencias Sociales. Programa de Investigación sobre Economía del Conocimiento en América Latina y el Caribe. Obtenido de \url{http://www.prosic.ucr.ac.cr/sites/default/files/documentos/caphum.pdf}

% 17
\bibitem{trejos-1}
Ignacio Trejos, Alvaro Cordero. 2017. \emph{Learn-by-doing-collaboratively across the curriculum: Integrative projects at UCenfotec}. IEEE World Engineering Education Conference – EDUNINE2017. 3.

% 18
\bibitem{salazar}
Gabriela Salazar. 2012 \emph{Desafíos del curso de ingeniería de software}. Revista de Educación en Ingeniería. Enero a Junio de 2012, Vol 7, Num 13. Páginas 32-43.

% 19
\bibitem{mora-et-al-1}
Sonia Mora, Mayela Coto, Georges Alfaro. 2014. \emph{A proposal for implementing PBL in programming courses}. Computing Conference (CLEI), 2014 XL Latin American. Páginas 1-11. DOI: \url{https://ezproxy.itcr.ac.cr:2878/10.1109/CLEI.2014.6965195}

% 20
\bibitem{mora-et-al-2}
Sonia Mora, Mayela Coto, Georges Alfaro. 2014. \emph{Giving more autonomy to computer engineering students: Are we ready?}. 2013 IEEE Global Engineering Education Conference (EDUCON). Páginas 618-626. DOI: \url{https://ezproxy.itcr.ac.cr:2878/10.1109/EduCon.2013.6530170} 

% 21
\bibitem{rashina-et-al}
Rashina~Hoda, Philippe~Kruchten, James~Noble y Stuart~Marshall. 2010. \emph{Agility in context}. SIGPLAN Not. 45, 10 (Octubre 2010), 74-88. DOI: \url{https://ezproxy.itcr.ac.cr:2878/10.1145/1932682.1869467}

% 22
\bibitem{ford}
Neil~Ford. 2011 \emph{Agile Engineering Practices}. Video. O`Reilly Media Inc. Obtenido de \url{https://www.safaribooksonline.com/library/view/neal-ford-on/9781449314439/}. ISBN: 9781449314439

% 23
\bibitem{diebold-dahlem}
Philipp~Diebold y Marc~Dahlem. 2014. \emph{Agile practices in practice: a mapping study}. En Proceedings of the 18th International Conference on Evaluation and Assessment in Software Engineering (EASE `14). ACM, New York, NY, USA, , Artículo 30 , 10 páginas. DOI: \url{http://ezproxy.itcr.ac.cr:2075/10.1145/2601248.2601254}

%24
\bibitem{shore-warden}
James~Shore y Shane Warden. \emph{The Art of Agile Development}. 2008. O'Reilly Media, Inc.

% 25
\bibitem{beck-andres}
Kent~Beck, Cynthia Andres. 2004. \emph{Extreme Programming Explained}. Addison-Wesley Professional.

% 26
\bibitem{bang-et-al}
Soon K. Bang, Sam Chung, Young Choh, y Marc Dupuis. 2013. \emph{A grounded theory analysis of modern web applications: knowledge, skills, and abilities for DevOps}. En Proceedings of the 2nd annual conference on Research in information technology (RIIT `13). ACM, New York, NY, USA, 61-62. \url{DOI=http://ezproxy.itcr.ac.cr:2075/10.1145/2512209.2512229}

% 27
\bibitem{anderson}
Charles~Anderson. 2015. \emph{Docker}. IEEE Software 32, 3 (May-June 2015). 102-c3. 4 páginas.  DOI: \url{https://doi.org/10.1109/MS.2015.62}.

% 28
\bibitem{jabbari-et-al}
Ramtin~Jabbari, Nauman bin Ali, Kai~Petersen y Binish Tanveer. 2016. \emph{What is DevOps?: A Systematic Mapping Study on Definitions and Practices}. En Proceedings of the Scientific Workshop Proceedings of XP2016 (XP `16 Workshops). ACM, New York, NY, USA, Artículo 12 , 11 páginas. DOI: \url{http://ezproxy.itcr.ac.cr:2075/10.1145/2962695.2962707} 

% 29
\bibitem{wiggins}
Adam~Wiggins. \emph{The Twelve-Factor App}. 2017. Obtenido de \url{https://12factor.net}

% 30
\bibitem{puppet-devops}
Puppet Labs, DevOps Research \& Assessment(DORA). 2017. \emph{State of DevOps}. Reporte. Obtenido de \url{https://puppet.com/resources/whitepaper/state-of-devops-report} 


% 31
\bibitem{kropp-meier-1}
Martin~Kropp, Andreas Meier. 2013. \emph{Teaching Agile Software Development at University Level: Values, Management, and Craftsmanship}. En Software Engineering Education and Training (CSEE\&T). Páginas 179 - 188. DOI: \url{https://ezproxy.itcr.ac.cr:2878/10.1109/CSEET.2013.6595249}

% 32
\bibitem{radermacher-walia}
Alex Radermacher and Gursimran Walia. 2013. \emph{Gaps between industry expectations and the abilities of graduates}. In Proceeding of the 44th ACM technical symposium on Computer science education (SIGCSE '13). ACM, New York, NY, USA, 525-530. DOI: \url{http://ezproxy.itcr.ac.cr:2075/10.1145/2445196.2445351}

% 33
\bibitem{murillo-trejos}
Jorge Murillo, Ignacio Trejos. 2017. \emph{IT Skills Mapping Engine}. In World Engineering Education Conference (EDUNINE), IEEE. Vol 2, March 29-22, 2017. Páginas 49-50


% 34
\bibitem{betz-et-al}
Charles Betz, Amos O. Olagunju, and Patrick Paulson. 2016. \emph{The Impacts of Digital Transformation, Agile, and DevOps on Future IT curricula}. In Proceedings of the 17th Annual Conference on Information Technology Education (SIGITE '16). ACM, New York, NY, USA, 106-106. DOI: \url{http://ezproxy.itcr.ac.cr:2075/10.1145/2978192.2978205}

% 35
\bibitem{acm-curriculum}
ACM. 2017. \emph{Curricula Recommendations}. Obtenido de: \url{http://www.acm.org/education/curricula-recommendations}

% 36
\bibitem{acm-cs-curriculum}
Joint Task Force on Computing Curricula, Association for Computing Machinery (ACM) and IEEE Computer Society. 2013. \emph{Computer Science Curricula 2013: Curriculum Guidelines for Undergraduate Degree Programs in Computer Science}. ACM, New York, NY, USA. DOI: \url{http://dx.doi.org/10.1145/2534860}

% 37
\bibitem{acm-se-curriculum}
The Joint Task Force on Computing Curricula. 2015. \emph{Curriculum Guidelines for Undergraduate Degree Programs in Software Engineering}. Technical Report. ACM, New York, NY, USA. 

% 38
\bibitem{acm-msis-curriculum}
Heikki Topi, Helena Karsten, Sue A. Brown, João Alvaro Carvalho, Brian Donnellan, Jun Shen, Bernard C. Y. Tan, and Mark F. Thouin. 2017. \emph{MSIS 2016: Global Competency Model for Graduate Degree Programs in Information Systems}. Technical Report. ACM, New York, NY, USA.

% 39
\bibitem{cenfotec-1}
Cenfotec 2011. \emph{Educación de calidad para la industria de software}. ALTEC 2001, IX Seminario Latino-Iberoamericano de Gestión Tecnológica. Innovación Tecnológica en la Economía del Conocimiento, Costa Rica, 10. 2001.


\bibitem{humble}
Jez Humble, David Farley. 2011. \emph{Continuous delivery: reliable software releases through build, test, and deployment automation}. Pearson Education, Inc.

\bibitem{michael-b}
Michael Hüttermann. 2012. \emph{DevOps for Developers}. Apress.

\end{thebibliography}

% biography section
% 
% If you have an EPS/PDF photo (graphicx package needed) extra braces are
% needed around the contents of the optional argument to biography to prevent
% the LaTeX parser from getting confused when it sees the complicated
% \includegraphics command within an optional argument. (You could create
% your own custom macro containing the \includegraphics command to make things
% simpler here.)
%\begin{IEEEbiography}[{\includegraphics[width=1in,height=1.25in,clip,keepaspectratio]{mshell}}]{Michael Shell}
% or if you just want to reserve a space for a photo:

\begin{IEEEbiography}[{\includegraphics[width=1in,height=1.25in,clip,keepaspectratio]{martin-flores}}]{Martín Flores}
es Ingeniero en Informática de la Universidad Nacional. Actualmente, realiza sus estudios de Maestría en Ciencias de la Computación del Tecnológico de Costa Rica. Sus principales intereses son: lenguajes de programación, ingeniería de software y \emph{DevOps}.
\end{IEEEbiography}


% You can push biographies down or up by placing
% a \vfill before or after them. The appropriate
% use of \vfill depends on what kind of text is
% on the last page and whether or not the columns
% are being equalized.

%\vfill

% Can be used to pull up biographies so that the bottom of the last one
% is flush with the other column.
%\enlargethispage{-5in}



% that's all folks
\end{document}


